\documentclass[12pt]{article}
\special{papersize=3in,5in}
\usepackage[utf8]{inputenc}
%PACKAGES
\usepackage{CJKutf8}
\usepackage[colorlinks = true,
            linkcolor = blue,
            urlcolor  = black,
            citecolor = blue,
            anchorcolor = blue]{hyperref}
\usepackage[T1]{fontenc}
\makeatletter
\def\ps@pprintTitle{%
  \let\@oddhead\@empty
  \let\@evenhead\@empty
  \let\@oddfoot\@empty
  \let\@evenfoot\@oddfoot
}
\usepackage{amssymb,amsmath,physics,amsthm,xcolor,graphicx}
\usepackage[shortlabels]{enumitem}
\newtheorem{observation}{Observation}
\newtheorem{theorem}{Theorem}
\newtheorem{proposition}{Proposition}
\newtheorem{lemma}{Lemma}
\newtheorem{definition}{Definition}
\newtheorem{corollary}{Corollary}
\newcommand{\red}[1]{{\color{red}#1}}
\usepackage[colorlinks = true,
            linkcolor = blue,
            urlcolor  = black,
            citecolor = blue,
            anchorcolor = blue]{hyperref}
\usepackage{cleveref}
\bibliographystyle{elsarticle-num}
\newcommand{\A}{\mathbb{A}}\newcommand{\C}{\mathbb{C}}\newcommand{\E}{\mathbb{E}}\newcommand{\F}{\mathbb{F}}\newcommand{\K}{\mathbb{K}}\newcommand{\LL}{\mathbb{L}}\newcommand{\M}{\mathbb{M}}\newcommand{\N}{\mathbb{N}}\newcommand{\PP}{\mathbb{P}}\newcommand{\Q}{\mathbb{Q}}\newcommand{\R}{{\mathbb R}}\newcommand{\T}{{\mathbb T}}\newcommand{\Z}{{\mathbb Z}}
\newcommand{\Aa}{\mathcal{A}}\newcommand{\Bb}{\mathcal{B}}\newcommand{\Cc}{\mathcal{C}}\newcommand{\Ee}{\mathcal{E}}\newcommand{\Ff}{\mathcal{F}}\newcommand{\Gg}{\mathcal{G}}\newcommand{\Hh}{\mathcal{H}}\newcommand{\Kk}{\mathcal{K}}\newcommand{\Ll}{\mathcal{L}}\newcommand{\Mm}{\mathcal{M}}\newcommand{\Nn}{\mathcal{N}}\newcommand{\Pp}{\mathcal{P}}\newcommand{\Qq}{\mathcal{Q}}\newcommand{\Rr}{{\mathcal R}}\newcommand{\Ss}zzzzz\newcommand{\Tt}{{\mathcal T}}\newcommand{\Zz}{{\mathcal Z}}\newcommand{\Uu}{{\mathcal U}}\newcommand{\Ww}{{\mathcal W}}

\newcommand\restr[2]{{\left.\kern-\nulldelimiterspace #1\vphantom{\big|} \right|_{#2}}}
 \newcommand{\br}[1]{\left\langle#1\right\rangle}
 \newcommand{\md}{\,\mmd}
 \newcommand{\wt}[1]{\widetilde{#1}}
 \newcommand{\ol}[1]{\overline{#1}}
 \newcommand{\wh}[1]{\widehat{#1}}
\pagestyle{empty}
\setlength{\parindent}{0in}

\begin{document}
\begin{CJK*}{UTF8}{gbsn}
\title{Some semigroup theory}
\author{Liam Llamazares}
\date{date}
\maketitle
\section{ Three line summary}
\begin{itemize}
\item Semigroups evolve naturally in PDE and act on the initial data evolving it forward in time. 
\item Semigroups come in various flavors and have generators. 
\item Semigroups help with this.
\end{itemize}
\section{Why should I care?}
The prototypical example of a semigroup is the one for the heat equation. It gives un an explicit solution. For other equations semigroup also play important roles and can even be used to prove
(local) well posedness of the Navier Stokes equations.
\section{Notation}

\section{Introduction}
Lets start simple, suppose we are given an ODE of the form 
\begin{equation*}
    u'=Au.
\end{equation*}
Where $A\in \R^{d\times d}$ is some constant matrix, then the solution to the above equation is 
\begin{equation*}
  u(t)=e^{At}.
\end{equation*}
If we now consider the slighly more complex $u'=Au+f$ where  $f\in C^1(\R)$ we get that
\begin{equation*}
  u(t)=e^{tA}u_0+\int_{0}^t e^{(t-s)A}f(s) ds.
\end{equation*}    
Heuristically, the first term represents the contribution of the initial data to our solution, whereas the second represents how the ``forcing term" $f$ makes it evolve in time. By setting $\Phi(t):=e^{At}$,
the above equation can also be rewritten as 
\begin{equation*}
    u(t)=\Phi(t)u_0+\int_{0}^t\Phi(t-s)f(s) ds.
\end{equation*}
   We will say that the above expression is the semigroup representation of $u$ and call $\Phi$ the semigroup of $u$ (or the differential equation that gives $u$). As we can see, 
   the semigroup determines the evolution of $f$. In the case where we have a linear ODE the semigroup is just a function $\Phi:\R\to\R$. However, in the case where our 
    equation is a PDE (also known as an infinite dimensional PDE), the semigroup will be a function 
    \begin{equation*}
        \Phi:\R_+\to L(H).
    \end{equation*}
        Where $H$ is the Hilbert space our solution lives  (typically a Soboleb space such as $H^1(\R^d)$) and  $L(H)$ is the space of linear endormorphisms on $H$.:wq




If now we make $A(t)$ depend on time we get that, by Duhamel's formula 
\begin{equation*}
  u(t)=e^{\int_{0}^tA(s)}u_0+\int_{0}^t e^{\int_{s}^t A(r)dr}f(s) ds=\Phi(t,0)u_0+\int_{0}^t \Phi(t,s)f(s)ds.
\end{equation*}



 \begin{equation*}
    .
\end{equation*}
    
    

\end{CJK*}



\bibliography{biblio.bib}
\end{document}

