\documentclass[12pt]{article}
\special{papersize=3in,5in}
\usepackage[utf8]{inputenc}
%PACKAGES
\usepackage[colorlinks = true,
    linkcolor = blue,
    urlcolor  = black,
    citecolor = blue,
    anchorcolor = blue]{hyperref}
\usepackage[T1]{fontenc}
\makeatletter
\def\ps@pprintTitle{%
    \let\@oddhead\@empty
    \let\@evenhead\@empty
    \let\@oddfoot\@empty
    \let\@evenfoot\@oddfoot
}
\usepackage{amssymb,amsmath,physics,amsthm,xcolor,graphicx}
\usepackage[shortlabels]{enumitem}
\newtheorem{observation}{Observation}
\newtheorem{theorem}{Theorem}
\newtheorem{proposition}{Proposition}
\newtheorem{lemma}{Lemma}
\newtheorem{definition}{Definition}
\newtheorem{corollary}{Corollary}
\newcommand{\red}[1]{{\color{red}#1}}
\usepackage[colorlinks = true,
    linkcolor = blue,
    urlcolor  = black,
    citecolor = blue,
    anchorcolor = blue]{hyperref}
\usepackage{cleveref}
\bibliographystyle{elsarticle-num}
\newcommand{\A}{\mathbb{A}}\newcommand{\C}{\mathbb{C}}\newcommand{\E}{\mathbb{E}}\newcommand{\F}{\mathbb{F}}\newcommand{\K}{\mathbb{K}}\newcommand{\LL}{\mathbb{L}}\newcommand{\M}{\mathbb{M}}\newcommand{\N}{\mathbb{N}}\newcommand{\PP}{\mathbb{P}}\newcommand{\Q}{\mathbb{Q}}\newcommand{\R}{{\mathbb R}}\newcommand{\T}{{\mathbb T}}\newcommand{\Z}{{\mathbb Z}}
\newcommand{\Aa}{\mathcal{A}}\newcommand{\Bb}{\mathcal{B}}\newcommand{\Cc}{\mathcal{C}}\newcommand{\Ee}{\mathcal{E}}\newcommand{\Ff}{\mathcal{F}}\newcommand{\Gg}{\mathcal{G}}\newcommand{\Hh}{\mathcal{H}}\newcommand{\Kk}{\mathcal{K}}\newcommand{\Ll}{\mathcal{L}}\newcommand{\Mm}{\mathcal{M}}\newcommand{\Nn}{\mathcal{N}}\newcommand{\Pp}{\mathcal{P}}\newcommand{\Qq}{\mathcal{Q}}\newcommand{\Rr}{{\mathcal R}}\newcommand{\Tt}{{\mathcal T}}\newcommand{\Zz}{{\mathcal Z}}\newcommand{\Uu}{{\mathcal U}}
\newcommand\restr[2]{{\left.\kern-\nulldelimiterspace #1\vphantom{\big|} \right|_{#2}}}
\newcommand{\br}[1]{\left\langle#1\right\rangle}
\pagestyle{empty}
\setlength{\parindent}{0in}

\begin{document}
\title{The Skorohod Integral}
\author{Liam Llamazares}
\date{05-28-2022}
\maketitle
\section{ Three line summary}
\begin{itemize}
	\item By fixing $t$, one can obtain a chaos expansion for (possibly non-adapted) square integrable stochastic processes $X(t)$.
	\item The Itô integral of an Itô integrable process $X(t)$ has a chaos expansion.
	\item This chaos expansion can converge even when $X(t)$ is not adapted to the filtration $\Ff_t$ (and thus not Itô integrable). This allows us to extend the Itô integral to non-adapted processes.
\end{itemize}
\section{Why should I care?}
If you want to define the Malliavin derivative you need the Skorohod integral.
\section{Notation}
The same as in the previous post \cite{Chaosblog} on the chaos expansion. We will also write $\mathbb{L}^2(I\times \Omega)$ for the space of Itô integrable functions (this was defined in \cite{Itointblog}).
\section{The chaos expansion of an Itô integral}
Our goal in this post is to construct the Skorohod integral. This serves as a generalization of the Itô integral and the starting point for the definition of the Malliavin derivative. How is this done? Let us first consider a (not-necessarily adapted) stochastic process $X$ such that $X_t\in L^2(\Omega,\Ff_\infty)$ for each $t \in I$. Then we know that by the chaos expansion proved in the previous post, for each $t$ there exists  $f_{n,t}\in L^2(S_n)$ such that
\begin{equation*}
	X=\sum_{n=0}^{\infty} I_n(f_{n,t}).
\end{equation*}
Let us write $f_{n}(\cdot,t):=f_{n,t}$. Note that we are now considering $f_n$ as a function of $n+1$ variables instead of  $n$. In particular, we will be able to consider expressions like  $I_{n+1}(f_n)$ later on. The first thing we do is study what the adaptedness of $X$ means in term of the functions $f_n$ appearing in its chaos expansion.
\begin{lemma}
	Let $X(t) \in L^2(\Omega,\Ff_\infty)$ for each $t \in  I$, then $X$ is adapted iff
	\begin{equation*}
		f_n(t_1,\ldots,t_n,t)=0,\quad\forall t\leq\max_{i=1,\ldots,n} t_i .
	\end{equation*}
\end{lemma}
\begin{proof}
	Firstly, we note that a stochastic process $X$ is adapted iff $$X_t=\E_{\Ff_t}[X(t)]\quad\forall t\in I.$$ Since the Itô integral is a martingale, we obtain that, by commuting the sum and using the uniqueness of the chaos expansion this is equivalent to requiring that, for all $t$
	\begin{multline*}
		I_n(f_n(\cdot,t))\\=	n!\E_{\Ff_t} \left[\int_{I} \left( \int_{0}^{t_n}\cdots \int_{0}^{t_2}f(t_1\ldots t_n,t) dW(t_1) \cdots dW(t_{n-1}) \right)dW(t_n)\right]\\
		=n!\int_{0}^t \int_{0}^{t_n}\cdots \int_{0}^{t_2}f(t_1\ldots t_n,t) dW(t_1) \cdots dW(t_{n-1})dW(t_n)\\= I_n(f_n(\cdot,t) 1_{\max_{t_i\leq t}})
	\end{multline*}
	Where the commutation of the sum and the integral is justified by the $L^2(\Omega)$ convergence of the chaos expansion ($L^1(\Omega)$ convergence would have been enough).
\end{proof}
In particular, we obtain that, since $f_n$ is already symmetric in its first n-coordinates, its symmetrization verifies that
\begin{equation*}
	f_{n,S}(t_1,\ldots,t_n,t_{n+1})=\frac{1}{n+1}f_n(t_1,\ldots\hat{t_{j}},\ldots,t_{n+1},t_j),\quad \text{where } j=\text{arg}\max_i t_i.
\end{equation*}
Using this relationship we can directly calculate the Itô integral of a stochastic process to obtain that.
\begin{theorem}
	Let $X \in \mathbb{L}^2(I\times\Omega)$ then the Itô integral of $X$ is
	\begin{equation*}
		\int_{I} X(t) dW(t)=\sum_{n=0}^{\infty} I_{n+1}(f_{n,S}).
	\end{equation*}
\end{theorem}
\begin{proof}
	This is a direct calculation using the previous result as


	\begin{multline*}
		\int_{I} X(t) dW(t)=\sum_{n=0}^{\infty}\int_{I} I_n(f_{n,t})dW(t)\\=\sum_{n=0}^{\infty}n! \int_{I}\int_{S_n}f_{n,t}(t_1,\ldots,t_n) dW(t_1)\ldots dW(t_n) dW(t)\\=\sum_{n=0}^{\infty}(n+1)! \int_{I}\int_{S_n}f_{n,S}(t_1,\ldots,t_n,t) dW(t_1)\ldots dW(t_n) dW(t)\\=\sum_{n=0}^{\infty}(n+1)! J_{n+1}(f_{n,S}) =\sum_{n=0}^{\infty}  I_{n+1}(f_{n,S}).
	\end{multline*}

\end{proof}
\section{The Skorohod integral}%

The last term appearing in the equality is what we will call the Skorohod integral.
\begin{definition}
	Let $X(t)\in L^2(\Omega,\Ff_\infty)$ be a  stochastic process such that
	\begin{equation*}
		\delta(X):=\int_{I} X(t)\delta W(t):=\sum_{n=0}^{\infty} I_{n+1}(f_{n,S})\in L^2(\Omega).
	\end{equation*}
	Then we will say that $X$ has Skorohod integral $\delta(X)$ and write $X\in dom(\delta)$.
\end{definition}
As we saw in the previous theorem the Skorohod integral is equal to the Itô integral for all stochastic processes in $\mathbb{L}^2(I\times\Omega)$. However, it may also be defined for non-adapted stochastic processes. In fact, by using the orthogonality of the iterated integrals (what we called \emph{Itô's} $n$-\emph{th isometry} in the last post \cite{Chaosblog}, we deduce the following).
\begin{proposition}
	A stochastic process $X(t)\in L^2(\Omega,\Ff_\infty)$ has a Skorohod integral iff
	\begin{equation*}
		\sum_{n=0}^{\infty} (n+1)!\|f_{n,S}\|^2_{L^2([0,T]^n)}<\infty.
	\end{equation*}

\end{proposition}
\begin{proof}
	By Itô's $n$-th isometry we have that
	\begin{equation*}
		\norm{\delta(X)}^2_{L^2(\Omega)}=\sum_{n=0}^{\infty}  \norm{I_{n+1}(f_{n,S})}_{L^2(\Omega)}^2=\sum_{n=0}^{\infty} (n+1)!\norm{f_{n,S}}_{L^2(I^{n+1})}^2.
	\end{equation*}
\end{proof}
Of course, a priori the above condition is not that easy to check for a given function as it involves calculating the chaos expansion for the given process $X$. In some cases however it is possible, to consider for example the stochastic process defined by $X(t)=W(T)$ on the interval $I=[0,T]$. Then we have that
\begin{equation*}
	X(t)=\int_{0}^T dW(t)=I_1(1).
\end{equation*}
Thus, for all $t\in I$ we have that
\begin{equation*}
	f_1=1;\quad f_n=0\quad\forall n \in \N \setminus \{1\}  .
\end{equation*}
So $X\in dom(\delta)$ with
\begin{equation*}
	\delta(X)= I_2(1)=\int_{0}^T\int_{0}^t dW(t_1)dW(t)=\int_{0}^T W(t) dW(t)= W^2(T)-T.
\end{equation*}
Note however that the Itô integral of $W(T)$ is undefined as it is not  $\Ff_t$ adapted. Since the Skorohod integral of $1$ is equal to  $W(T)$, the above example shows how one cannot simply ``pull out constants in $t$'' in the sense that, if $G$ is a random variable independent of  $t$ and  $X(t)=G\cdot u(t)$, then
\begin{equation*}
	\int_{I} X(t) \delta W(t)=\int_{I}G\cdot u(t)\delta W(t) \neq G\int_{I}u(t) dW(t).
\end{equation*}
Though this may seem unintuitive, it is a consequence of the fact that, even though $f_i$ may not depend on  $t$, the terms
\begin{equation*}
	g(t):=\int_{0}^t\int_{0}^{t_{n}}\cdots \int_{0}^{t_2} f_i dW(t_1)\cdots dW(t_{n-1})dW(t_n).
\end{equation*}
Can depend on $t$. Despite this, the Skorohod integral still maintains some of the natural properties we associate with integration.
\begin{proposition}
	Let $X(t), Y(t)\in dom(\delta)$, $\lambda \in \R$. Then it holds that
	\begin{itemize}
		\item $X(t)+ \lambda Y(t) \in dom(\delta)$ with $\delta(X+\lambda Y)=\delta(X)+\lambda \delta(Y)$.
		\item $\E[\delta(X)]=0$.
		\item $X\cdot 1_A \in dom(\delta)$ for any measurable subset $A \subset I$. Furthermore, if $A \cup B =I$ then
		      \begin{equation*}
			      \int_{A}X(t) \delta(t)+ \int_{B}X(t)\delta W(t):=\delta(X\cdot 1_A)+\delta(X\cdot 1_B)=\delta(X).
		      \end{equation*}

	\end{itemize}
\end{proposition}

\begin{proof}
	The first property is a consequence of the chaos expansion's linearity (which is itself  a consequence of the linearity of iterated Itô integration). The second is due to the expectation of the Itô integral being $0$. The final property is a consequence of the fact that the chaos expansion of $X\cdot 1_A$ is
	\begin{equation*}
		X\cdot 1_A=\sum_{n=0}^{\infty} I_{n}([f_n 1_A]_S).
	\end{equation*}
	Which shown by the equivalent characterization of Skorohod functions that $X\cdot 1_A\in dom(\delta)$. The final property is a consequence of the previously proved linearity.
\end{proof}
We now know what the Skorohod expansion is, how to characterize it, and its main properties, in the next post we will construct the Malliavin derivative as its adjoint.




\bibliography{biblio.bib}
Testing
\end{document}
嘿加奥,我看到Mihail来到了。我有有些中文的问题。如果你有时间以后可能我们可以聊天一下吗?
我也有一个开会一点时候直到二点半。所以如果你的会议到一点还没完了没事。我们可以改天再谈。
