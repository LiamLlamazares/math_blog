\documentclass[
    a4paper,
    DIV=14,
    abstract=true,
    numbers=noenddot
]
{scrartcl}

\usepackage{
    amsmath,
    amssymb,
    amsthm,
    array,
    authblk,
    bm,
    dsfont, % for 1 vector or indicator function
    graphicx,
    mathtools,
    nicefrac,
    physics,
    tabularx,
    tcolorbox,
    todonotes,
    tikz,
    xcolor,
    float,
}
\usepackage[shortlabels]{enumitem}

\usepackage[T1]{fontenc}

\usepackage[pdffitwindow=false,
    plainpages=false,
    pdfpagelabels=true,
    pdfpagemode=UseOutlines,
    pdfpagelayout=SinglePage,
    bookmarks=false,
    colorlinks=true,
    hyperfootnotes=false,
    linkcolor=blue,
    urlcolor=blue!30!black,
    citecolor=green!50!black]{hyperref}


\newtheorem{theorem}{Theorem}[section]
\newtheorem{proposition}[theorem]{Proposition}
\newtheorem{lemma}[theorem]{Lemma}
\newtheorem{corollary}[theorem]{Corollary}
\newtheorem{definition}[theorem]{Definition}

\theoremstyle{definition}
\newtheorem{example}[theorem]{Example}
\newtheorem{observation}{Observation}
\newtheorem{assumption}{Assumption}

\newtheorem{exercise}{Exercise}

\newtheorem*{hint}{Hint}

\newenvironment{exerciseandhint}[2]
{\begin{exercise} #1

    \emph{Hint: #2}}
    {\end{exercise}}

\newcommand{\red}[1]{{\color{red}#1}}
\newcommand{\td}{\todo[inline,color=green!40]}

\bibliographystyle{elsarticle-num}
\newcommand{\fk}[1]{\mathfrak{#1}}
\newcommand{\wh}[1]{\widehat{#1}}
\newcommand{\tl}[1]{\widetilde{#1}}

\newcommand{\br}[1]{\left\langle#1\right\rangle}
\newcommand{\set}[1]{\left\{#1\right\{}
\newcommand{\qp}[1]{\left(#1\right)}\newcommand{\qb}[1]{\left[#1\right]}
\newcommand{\qt}[1]{\left(#1\right)}
\newcommand{\Id}{\bm{I}}\renewcommand{\ker}{\bm{ker}}\newcommand{\supp}[1]{\bm{supp}(#1)}\renewcommand{\tr}[1]{\mathrm{tr}\left(#1\right)}
\renewcommand{\norm}[1]{\left\lVert #1 \right\rVert}\renewcommand{\abs}[1]{\left| #1 \right|}
\newcommand{\U}{}\renewcommand{\star}{*}
\renewcommand{\Im}{\bm{Im}}
\newcommand{\iso}{\xrightarrow{\sim}}
\renewcommand{\d}{\,\mathrm{d}}\newcommand{\dx}{\,\mathrm{d}x}
\newcommand{\dy}{\,\mathrm{d}y}
\newcommand\restr[2]{\left.#1\right|{#2}}
\newcommand{\rm}[1]{\mathrm{#1}}

\newcommand{\A}{\mathbb{A}}
\newcommand{\C}{\mathbb{C}}
\newcommand{\E}{\mathbb{E}}
\newcommand{\F}{\mathbb{F}}
\newcommand{\N}{\mathbb{N}}
\newcommand{\Q}{\mathbb{Q}}
\newcommand{\R}{\mathbb{R}}
\newcommand{\Z}{\mathbb{Z}}
\newcommand{\Aa}{\mathcal{A}}
\newcommand{\Bb}{\mathcal{B}}
\newcommand{\Cc}{\mathcal{C}}
\newcommand{\Dd}{\mathcal{D}}
\newcommand{\Ff}{\mathcal{F}}
\newcommand{\Gg}{\mathcal{G}}
\newcommand{\Hh}{\mathcal{H}}
\newcommand{\Kk}{\mathcal{K}}
\newcommand{\Ll}{\mathcal{L}}
\newcommand{\Mm}{\mathcal{M}}
\newcommand{\Nn}{\mathcal{N}}
\newcommand{\Oo}{\mathcal{O}}
\newcommand{\Pp}{\mathcal{P}}
\newcommand{\Qq}{\mathcal{Q}}
\newcommand{\Rr}{\mathcal{R}}
\newcommand{\Ss}{\mathcal{S}}
\newcommand{\Tt}{\mathcal{T}}
\newcommand{\Uu}{\mathcal{U}}
\newcommand{\Vv}{\mathcal{V}}
\newcommand{\Ww}{\mathcal{W}}
\newcommand{\Xx}{\mathcal{X}}
\newcommand{\Yy}{\mathcal{Y}}
\newcommand{\Zz}{\mathcal{Z}}

\begin{document}
\title{Fractional and negative order Sobolev spaces}
\author{Liam Llamazares}
\date{\today}
\maketitle
\section{Three line summary}
\begin{enumerate}
    \item There are three main ways to define fractional Sobolev spaces. Two of them, denoted by $W^{s,p},B^{s,p}$,  are defined by using the analogous to the definition of H\"older spaces and coincide when $s$ is not an integer. The second, denoted by $H^{s,p}$ , is defined by using the Fourier transform and coincides with $W^{s,p}$ for integer order regularity. Here $s$ denotes the (fractional) regularity and $p$ the integrability.
    \item All these spaces coincide with $H^s$  when $p=2$ is.
    \item The dual of $H^{s,p}(\R^d )$ is $H^{-s,p'}(\R^d )$. For integer order regularity and on smooth domains, the dual of $H^{k,p}(\Omega)$ is $H^{-k,p'}(\Omega)$ and can equivalently be defined by differentiating $k$ times  functions in $L^p(\R^d)$ . The analogous result holds for $W^{k,p}(\Omega)$ and $B^{k,p}(\Omega)$.
    \item Using these fractional estimates, one can obtain finer regularity results such as in the trace theorem.
\end{enumerate}
\subsection{Fractional Sobolev spaces: two definitions}
The definitions developed in the next three subsections can be found in \cite{agranovich2015sobolev} page 222.
\subsubsection{Soboelv-Slodeckij spaces}

\begin{definition}[Sobolev-Slodeckij spaces]\label{soledkij def}
    Let $s=k+\gamma$ where $k \in \N_0$, and $\gamma \in [0,1)$. Then, given  $p \in [1,\infty)$ and $\Omega  \subset \R^n$ be an arbitrary open set. Write $k=\left\lfloor s \right\rfloor,\gamma =k-s$.  We define
    \begin{align*}
        W^{s ,p}(U):= \set{u \in W^{k ,p}(U): \norm{u}_{W^{s,p}(U)}<\infty},
    \end{align*}
    where
    \begin{align}\label{norm def}
        \norm{u}_{W^{s,p}(U)}:= \left(\norm{u}_{W^{k,p}(U)}^p+ \sum_{\abs{\alpha}=k }\int_{U}\int_{U}\frac{\abs{D^\alpha u(x+y)-D^\alpha u(x)}^p}{\abs{y}^{n+\gamma p}}\d x \d y\right)^\frac{1}{p}.
    \end{align}
\end{definition}
We will later define $W^{s,p}(U)$ also for negative $s$ (see Definition \ref{negative s slodeckij}). We observe that the above definition coincides with our usual definition of Sobolev space when $s=k \in \N_0$ and  mimics that of the H\"older spaces, with the addition that we now require integrability. The factor $\abs{x-y}^{n+\gamma p}$ is chosen so that the integral is scale invariant.
\begin{exercise}
    Show that $W^{s,p}(U)$ is a Banach space.
\end{exercise}
\begin{hint}
    To show that $| \cdot |_{s,p}$ is a norm apply Minkowski's inequality to $u$ and to $f_u(x,y):=(u(x)-u(y))/(x-y)^{n/p+s}$. Given a Cauchy sequence show that, since $L^p(U)$ is complete, $u_n \to u$ in $L^p(U)$ and that $f_{u_n} \to f_u$ in $L^p(U\times U)$ to conclude that $u_n \to u$ in $W^{s,p}(U)$.
\end{hint}
Though the Sobolev-Slodeckij spaces can be defined for any open set $U$, they are most useful when $U=\R^d$ or $U$  is bounded open and Lipschitz (that is $U$ is of class $C^{0,1}$). This is because of the following result.
\begin{proposition}[Inclusion ordered by regularity]
    Let  $\Omega \subset \R^d$ be bounded open and Lipschitz. Then, for $p \in [1,\infty)$ and $0<s<s'$ it holds that
    \begin{align*}
        W^{s',p}(\Omega )\hookrightarrow W^{s,p}(\Omega ), \quad W^{s',p}(\R^d)\hookrightarrow W^{s,p}(\R^d).
    \end{align*}
\end{proposition}
The proof can be found in \cite{di2012hitchhiker's} page 10. The regularity of the domain is necessary to be able to extend functions in $W^{1,p}(\Omega )$ to $W^{1,p}(\R^d)$. The result is not true otherwise and an example is given in this same reference.
\subsubsection{Bessel potential spaces}
We now give a second definition of fractional Sobolev spaces through the Fourier transform.
\begin{definition}[Bessel potential spaces on $\R^d$ ]\label{bessel potential def}
    Let $s>0$ and $p \in [1,\infty)$. Define for $u \in \Ss'(\R^d)$
    \begin{align*}
        \Lambda^s u := \Ff^{-1}\left(\br{\xi}^s \wh{u}(\xi)\right).
    \end{align*}

    Then, we define the \emph{Bessel potential space}
    \begin{align*}
        H^{s,p}(\R^d):=\left\{u \in \mathcal{S}^{\prime}(\mathbb{R}^d): \Lambda ^s u \in L^p(\mathbb{R}^d)\right\},
    \end{align*}
    and give it the norm
    \begin{align*}
        \norm{u}_{H^{s,p}(\R^d)}:= \norm{\Lambda^s u}_{L^p(\R^d)}.
    \end{align*}
    We also define the space $H_0^{s,p}(\R^d)$ as the closure of $C_c^\infty(\R^d)$ in $H^{s,p}(\R^d)$.
\end{definition}
In the definition above, is motivated by the case $p=2$. As we saw when we studied Sobolev spaces through the \href{https://nowheredifferentiable.com/2023-01-29-PDE-1-Fourier/#:~:text=Sobolev%20spaces-,Sobolev%20spaces,-form%20a%20particular}{Fourier transform}, $u \in H^k(\R^d)$ if and only if $\Lambda^k u \in L^2(\R^d)$. That is, $H^{k,2}(\R^d)=H^{k}(\R^d)$.
The natural generalization of this fact gives Definition \ref{bessel potential def}.
\begin{exercise}
    Show that $\Lambda^s\Lambda^r=\Lambda^{s+r}$. Use this to show that
    \begin{align*}
        \Lambda^r: H^{r+s,p}(\R^d) \iso  H^{s,p}(\R^d),
    \end{align*}
    is an invertible isomorphism.
\end{exercise}
\begin{hint}
    Use that $\br{\xi}^s\br{\xi}^r=\br{\xi}^{s+r}$ and show that the inverse of $\Lambda^r$ is $\Lambda^{-r}$.
\end{hint}


We now extend this to general domains
\begin{definition}[Bessel potential spaces on $U$]\label{bessel potential def U}
    Let $U \subset \R^d$ be an arbitrary open set. We define,
    \begin{align*}
        H^{s,p}(U):=\left\{u \in \mathcal{D}^{\prime}(U): \text{ there exists } v \in H^{s,p}(\R^d) \text{ such that } \restr{v}{U}=u\right\}.
    \end{align*}
    And give it the norm
    \begin{align*}
        \norm{u}_{H^{s,p}(U)}:= \inf \set{\norm{v}_{H^{s,p}(\R^d)}: \restr{v}{U}=u}.
    \end{align*}
\end{definition}
The restriction above is in the sense of distributions. That is, we define $u=\restr{v}{U}$ by
\begin{align*}
    (\phi,u):=(\phi,v), \quad \forall \phi \in C_c^\infty(U).
\end{align*}

It would be tempting to define $\norm{u}_{H^{s,p}(U)}:=\norm{\Lambda^s v}_{L^p(U)}$. However, since the Fourier transform, and thus $\Lambda^s$, is a nonlocal operator, the norm would depend on the extension $v$  of $u$ to $\R^d$. So the norm would be ill-defined.
\subsubsection{Besov spaces}
\begin{definition}[Besov spaces]\label{besov def}
    Let $s=k_{-}+\gamma$ where $k^{-} \in \N_0$, and $\gamma \in (0,1]$. Then, given  $p \in [1,\infty)$ and $\Omega  \subset \R^n$ be an arbitrary open set we define
    \begin{align*}
        B^{s ,p}(U):= \set{u \in W^{k^{-} ,p}(U): \norm{u}_{B^{s,p}(U)}<\infty},
    \end{align*}
    where
    \begin{align*}
        \norm{u}_{B^{s,p}(U)}:= \left(\norm{u}_{W^{k^{-},p}(U)}^p+ \sum_{\abs{\alpha}=k }\int_{U}\int_{U}\frac{\abs{D^\alpha u(x+y)-D^\alpha u(x)}^p}{\abs{y}^{n+\gamma p}}\d x \d y\right)^\frac{1}{p}.
    \end{align*}
\end{definition}
The above definition is extremely similar in form to that of the Sobolev-Slodeckij spaces \ref{soledkij def}. In fact, it is equivalent for $s \notin \N$.  The difference is that in the definition fo Besov spaces \ref{besov def} we require that $\gamma >0$. As a result, always $k^{-}<s$. We have chosen to indicate this fact by the index ``$-$'' on $k_{-}$. An equivalent definition is possible which extends the above to negative values of $s$
\begin{definition}[Besov spaces, negative $s$]
    Let $s \in \R$ and choose any $\sigma \not\in \N_0$ with $\sigma >0$. Then, given  $p \in [1,\infty)$ we define
    \begin{align*}
        \|u\|_{B^{s,p}(\Omega)}=\left\|\Lambda^{s-\sigma} u\right\|_{W_p^\sigma(\Omega)}.
    \end{align*}
\end{definition}
The requirement $\sigma >0$ is necessary as in general $B^{s,p}(\R^d)\neq H^{s,p}(\R^d)$
This can then be extended to general open sets $U$ in the same way as for the Bessel potential spaces,
\begin{definition}[Besov spaces on $U$]
    Let $U \subset \R^d$ be an arbitrary open set. We define,
    \begin{align*}
        B^{s,p}(U):=\left\{u \in \mathcal{D}^{\prime}(U): \text{ there exists } v \in B^{s,p}(\R^d) \text{ such that } \restr{v}{U}=u\right\},
    \end{align*}
    and give it the norm
    \begin{align*}
        \norm{u}_{B^{s,p}(U)}:= \inf \set{\norm{v}_{B^{s,p}(\R^d)}: \restr{v}{U}=u}.
    \end{align*}
\end{definition}
\begin{observation}
    Different authors use different notation for these spaces. For example, in \cite{triebel1992theory}, the notation $W^{s,p}:= B^{s,p}$ is used. With this notation, one has that, for $p \neq 2$,
    \begin{align*}
        H^{k,p} \neq B^{k,p}= W^{k,p}.
    \end{align*}
    Whereas, with our notation, as we will later see, $H^{k,p}=W^{k,p}$. Other notations which can be found are  the notation $B^{s,p}= \Lambda^{p}_s$ and $H^{s,p}= \Ll ^{p}_s$. See \cite{stein1970singular} and \cite{biccari2018local}.
\end{observation}
\subsubsection{Interpolation}
Both the Sobolev-Slobodeckij and Bessel potential spaces can be viewed as a way to fill the gaps between integer valued Sobolev spaces.
\begin{proposition}[Interpolation ]\label{interpolation}
    Let $s_0 \neq s_1 \in \R, p \in (1, \infty)$, $0<\theta<1$ and
    \begin{align*}
        s=s_0(1-\theta)+s_1 \theta, \quad p=p_0(1-\theta)+p_1 \theta.
    \end{align*}
    Then, given $\Omega \subset \R^d$ open with uniformly Lipschitz boundary it holds that
    \begin{align*}
        B^{s,p}=\left[B^{s_0,p}(\Omega ), B^{s_1, p}(\Omega )\right]_\theta,\quad H^{s,p}(\Omega )=\left[H^{s_0,p_0}(\Omega), H^{s_1,p_1}(\Omega)\right]_{\theta},
    \end{align*}
    where $[X,Y]_\theta$ denotes the complex interpolation space.
\end{proposition}
The result can be found in \cite{triebel1992theory} page 45 for $\Omega = \R^d$. The general results follows by extension. See \cite{leoni2017first} page 424. In particular, if we write $k:=\left\lfloor s \right\rfloor$ and $\gamma:=s-k$, then
\begin{align*}
    H^{s,p}(\Omega )=\left[H^{k,p}(\Omega), H^{k+1,p}(\Omega)\right]_{\gamma }= \left[L^p(\Omega ), H^{k+1,p}(\Omega\right]_{s/(k+1) }.
\end{align*}
\subsection{Relationship between the definitions}
The following result shows the inclusions between $W^{s,p},H^{s,p},B^{s,p}$ and can be found in \cite{agranovich2015sobolev} page 224 and in \cite{stein1970singular} page 155.
\begin{theorem}\label{equivalence fractional spaces}
    Let $\Omega \subset \R^d$ be open with uniformly Lipschitz boundary and $s \in [0,\infty)$. Then,
    \begin{align*}
        H^{s+\epsilon,p}(\Omega ) & \subset B^{s,p}(\Omega )  \subset H^{s,p}(\Omega )\quad \forall p \in (1,2]       \\
        B^{s+\epsilon,p}(\Omega ) & \subset H^{s,p}(\Omega )  \subset B^{s,p}(\Omega )\quad \forall p \in [2,\infty),
    \end{align*}
    where the above inclusions are continuous and dense. Furthermore,
    \begin{align}\label{slodeckij equivalence}
        W^{s,p}(\Omega )= \begin{cases}
                              H^{s,p}(\Omega ) & \text{ if } s \in \N_0    \\
                              B^{s,p}(\Omega ) & \text{ if } s \notin \N_0
                          \end{cases}.
    \end{align}
    In consequence, for $p=2$,
    \begin{align}\label{p=2}
        H^{s,2}(\Omega )=W^{s,2}(\Omega )=B^{s,2}(\Omega ).
    \end{align}
\end{theorem}
The equality in \eqref{slodeckij equivalence} shows that, as long as we understand the behaviour of $H^{s,p}(\Omega )$ and $B^{s,p}(\Omega )$ we can completely determine that of $W^{s,p}(\Omega )$. It also justifies the following extension of $W^{s,p}(\Omega )$ to negative regularity.
\begin{definition}[Slodeckij space negative $s$]\label{negative s slodeckij}
    Let $\Omega \subset \R^d$ be open with uniformly Lipschitz boundary. Then, given $p \in [1,\infty)$ and any  $s \in \R$ we define
    \begin{align*}
        W^{s,p}(\Omega )= \begin{cases}
                              H^{s,p}(\Omega ) & \text{ if } s \in \N_0    \\
                              B^{s,p}(\Omega ) & \text{ if } s \notin \N_0
                          \end{cases}.
    \end{align*}
\end{definition}
The equality for $p=2$ in \eqref{p=2} justifies that, for sufficiently regular domains, all three spaces are written $H^s(\Omega )$.   We will prove the left hand side of this equivalence in Exercise \ref{equivalence of fractional spaces}. For $p\neq 2$ the inclusions are in general strict. An example is constructed in \cite{stein1970singular} page 161 exercise 6.8.
\begin{exercise}[Equivalence of fractional spaces]\label{equivalence of fractional spaces}
    Show that
    \begin{align*}
        H^{s,2}(\R^d)=W^{s,2}(\R^d).
    \end{align*}
\end{exercise}
\begin{hint}
    We want to show that the norms are equivalent. That is, that
    \begin{align*}
        \norm{u}_{B^{s,2}(\R^d)}\sim \norm{u}_{H^{s,2}(\R^d)}.
    \end{align*}
    We already know this is the case when $s$ is an integer so it suffices to show that the norms are equivalent for $s= \gamma  \in (0,1)$. That is, that
    \begin{align*}
        |u|_{s,2}^2\sim \int_{\mathbb{R}^d}|\xi|^{2 s}|\mathcal{F} u(\xi)|^2 d \xi
    \end{align*}
    By multiple changes of variable and Plancherel's theorem we have that
    \begin{align*}
         & |u|_{\gamma ,2}^2  =\int_{\R^d}\int_{\R^d}\frac{\abs{u(x+y)-u(y)}^2}{\abs{x}^{d+2\gamma	}}\d x \d y                                                                                                       = \int_{\R^d}\frac{\norm{\Ff (u(x+\cdot )-u)}^2}{\abs{x}^{d+2\gamma	}}\d x \\
         & =\int_{\R^d}\int_{\R^d}  \frac{|e^{-2 \pi i x \cdot \xi}-1|^2}{\abs{x}^{d+2\gamma	}}|\wh{u}(\xi)|^2\d x\d\xi =\int_{\R^d}\left(\int_{\R^d}  \frac{1-\cos(2\pi \xi\cdot x)}{\abs{x}^{d+2\gamma	}}\d x\right)|\wh{u}(\xi)|^2\d\xi.
    \end{align*}
    To treat the inner integral we note that it is rotationally invariant and so, by rotating, to the first axis and later changing variable $x \to x / \abs{\xi}$ we get
    \begin{align*}
        \int_{\R^d}  \frac{1-\cos(2\pi \xi\cdot x)}{\abs{x}^{d+2\gamma	}}\d x & =\int_{\R^d}  \frac{1-\cos(2\pi \abs{\xi}x_1 )}{\abs{x}^{d+2\gamma	}}\d x                                          \\
                                                                              & =\abs{\xi}^{2 \gamma } \int_{\R^d}  \frac{1-\cos(2\pi  x_1) }{\abs{x}^{d+2\gamma	}}\d x\sim \abs{\xi}^{2 \gamma }.
    \end{align*}
    The last integral is finite as, since $d+2\gamma >d$, the tails $\abs{\xi}\to\infty$ are controlled, and since $1-\cos(2\pi x_1)\sim x_1^2\leq \abs{x}^2$ the integrand has order $-d+2(1-\gamma)>-d$ for $\abs{\xi}\sim 0$ . That said, substituting this back into the previous expression gives the desired result.
\end{hint}
\begin{exercise}
    Use the previous exercise \ref{equivalence of fractional spaces} to show that, if $\Omega $ is a open set with uniformly Lipschitz boundary, then
    \begin{align*}
        H^{s,2}(\Omega )=W^{s,2}(\Omega ).
    \end{align*}
\end{exercise}
\begin{hint}
    By definition \ref{bessel potential def} choose a sequence $v_n \in H^{s,2}(\R^d)$ such that $\norm{v_n}_{H^{s,2}(\R^d)} \to \norm{u}_{H^{s,2}(\Omega )}$ in $H^s(\Omega )$. Then,
    \begin{align*}
        \norm{u}_{H^{s,2}(\Omega)}= \lim_{n\to\infty}\norm{v_n}_{H^{2,2}(\R^d)}\sim \lim_{n\to\infty}\norm{v_n}_{B^{s,2}(\R^d)}\geq \norm{u}_{B^{s,2}(\Omega )}.
    \end{align*}
    To obtain the reverse inequality use the existence of a continuous extension operator $E:W^{s,2}(\Omega )\to W^{s,2}(\R^d)$ (see \cite{di2012hitchhiker's} page 33) to obtain
    \begin{align*}
        \norm{u}_{W^{s,2}(\Omega )}\sim \norm{Eu}_{W^{s,2}(\R^d)}\geq \norm{u}_{H^{s,2}(\R^d)}.
    \end{align*}

\end{hint}

The above suggests that integrals appearing in the definition of the slodeckij spaces \ref{soledkij def} correspond to differentiating a fractional amount of times. This indeed is the case
\begin{definition}
    Given $\gamma  \in [0,+\infty)$ and $u \in \Ss (\R^d)$ we define the fractional Laplacian as
    \begin{align*}
        (-\Delta )^{\gamma }u(x):= \mathcal{F}^{-1}(\abs{2\pi\xi}^{2\gamma }\wh{u}(\xi )).
    \end{align*}
\end{definition}
\begin{proposition}
    For $\gamma  \in (0,1)$ and $u \in H^{s}(\R^d)$ it holds that
    \begin{align*}
        (-\Delta )^{\gamma }u(x)=C\int_{\R^d}\frac{u(x)-u(x+y)}{\abs{y}^{d+2\gamma}}\d y,
    \end{align*}
    where $C$ is a constant that depends on $d,\gamma $.
\end{proposition}
\begin{proof}
    The above equality may seem odd at first if we compare with the integral in \ref{soledkij def} where a square appears in the numerator which gives us our $2$  in the $2 \gamma $. However, it is justified by the fact that, by the change of variables $y \to -y$,
    \begin{align*}
        \int_{\R^d}\frac{u(x)-u(x+y)}{\abs{y}^{d+2\gamma}}\d y=\int_{\R^d}\frac{u(x)-u(x-y)}{\abs{y}^{d+2\gamma}}\d y.
    \end{align*}
    So we can get the \emph{second} order difference in the numerator by adding the two integrals.
    \begin{align}\label{second order}
        \int_{\R^d}\frac{u(y)-u(x+y)}{\abs{y}^{d+2\gamma}}\d y=-\frac{1}{2}\int_{\R^d}\frac{u(x+y)-2u(x)+u(x-y)}{\abs{y}^{d+2\gamma}}\d y.
    \end{align}
    That said, we must show that
    \begin{align*}
        \abs{\xi}^{2\gamma }\wh{u}(\xi )\sim \Ff \left(\int_{\R^d}\frac{u(x)-u(x+y)}{\abs{y}^{d+2\gamma}}\d y\right)
    \end{align*}
    Using \eqref{second order} and proceeding as in exercise \ref{equivalence fractional spaces} gives
    \begin{align*}
         & \Ff \left(\int_{\R^d}\frac{u(x)-u(x+y)}{\abs{y}^{d+2\gamma}}\d y\right)= -\frac{1}{2} \int_{\R^d}\left(\int_{\R^d}\frac{e^{-2\pi i y \cdot \xi}-2+e^{2\pi i y \cdot \xi}}{\abs{y}^{d+2\gamma}}\d y\right) \wh{u}(\xi)\d \xi       \\
         & =\int_{\R^d}\left(\int_{\R^d}\frac{1-\cos(2\pi y \cdot \xi)}{\abs{y}^{d+2\gamma}}\d y\right) \wh{u}(\xi)\d \xi =\int_{\R^d}  \frac{1-\cos(2\pi  y_1) }{\abs{y}^{d+2\gamma	}}\d y\int_{\R^d}\abs{\xi}^{2 \gamma }\wh{u}(\xi)\d \xi \\& \sim \abs{\xi}^{2 \gamma }\wh{u}(\xi)\d \xi.
    \end{align*}
    This completes the proof, and shows that the explicit expression for $C$ is
    \begin{align*}
        C=\frac{1}{(2\pi)^{2 \gamma }}\int_{\R^d}  \frac{1-\cos(2\pi  y_1) }{\abs{y}^{d+2\gamma	}}\d y.
    \end{align*}
\end{proof}
\section{Dual of Sobolev spaces and correspondence with negative regularity}
Negative orders of regularity correspond to the dual of Sobolev spaces. This is best seen in the integer case, where the following result holds (see \cite{evans2022partial} pages (326-344) for the case $p=2$ ).
\begin{theorem}\label{dual of integer sobolev}
    For all $k \in \mathbb{Z}$ and $p \in [1,\infty)$ it holds that
    \begin{align*}
        H^{k,p}_0(\Omega )' = H^{-k,p'}(\Omega ), \quad W^{k,p}_0(\Omega )' = W^{-k,p'}(\Omega ).
    \end{align*}
\end{theorem}
The first equality will be discussed in the next subsection and is most easily proves when $\Omega =\R^d$, in which case one can use the correspond $\Lambda ^s : H^{r,p}(\R^d )\iso H^{r-s,p}(\R^d )$ together with the reflexivity of $L^p(\R^d )$. The second equality is a direct consequence of the integer order equality $W^{k,p}(\Omega )=H^{k,p}(\Omega )$ of Theorem \ref{equivalence fractional spaces}. For fractional order regularities, we have the following result which can be found in \cite{agranovich2015sobolev} page 228.
\begin{theorem}
    The spaces $W^{s,p}(\Omega ),H^{s,p}(\Omega ),B^{s,p}(\Omega )$ are reflexive Banach spaces with duals
    \begin{align*}
        W^{s,p}(\Omega )'=	W^{-s,p'}_{\overline{\Omega } }(\R^d ), \quad H^{s,p}(\Omega )' = H^{-s,p'}_{\overline{\Omega } }(\R^d ), \quad B^{s,p}(\Omega )' = B^{-s,p'}_{\overline{\Omega } }(\R^d ).
    \end{align*}
    where $p'$ is the conjugate exponent of $p$ and given a space of distributions $X$ on $\R^d$ we define $X_{\overline{\Omega }}$ as the space of distributions on $\R^d$ which are supported in $\overline{\Omega }$. In particular, for $\Omega =\R^d$,
    \begin{align*}
        W^{s,p}(\R^d )'=	W^{-s,p'}(\R^d ), \quad H^{s,p}(\R^d )' = H^{-s,p'}(\R^d ), \quad B^{s,p}(\R^d )' = B^{-s,p'}(\R^d ).
    \end{align*}
\end{theorem}
We have already seen that $W^{s,p}(\Omega )$ and $H^{s,p}(\Omega )$ are Banach spaces, one can similarly show that $B^{s,p}(\Omega )$ is a Banach space. Furthermore, the spaces are all reflexive for smooth domains.
\begin{observation}
    Some authors define given $s>0$
    \begin{align*}
        W^{-s,p'}(\Omega )':=	W^{s,p}(\Omega )'.
    \end{align*}
    See for example \cite{biccari2018local}. This is equivalent to our definition when $\Omega = \R^d$ or when $s \in k$. However, in other cases, the two definitions are not equivalent.
\end{observation}

\subsection{The dual of $H^{s,p}(\R^d)$ and $B^{s,p}(\R^d)$}
For some motivation we start by considering the case $\Omega =\R^d$. In this case, since the closure of
$C_c^\infty(\R^d)$ in $H^{s,p}(\R^d)$ (which by definition is $H_0^{s,p}(\R^d)$) is itself $H^{s,p}(\R^d)$, we have that $H_0^{s,p}(\R^d)=H^{s,p}(\R^d)$.
\begin{exercise}[Dual identification]\label{dual exercise}
    Prove the identification $H^{-s,p'}(\R^d)=H^{s,p}(\R^d)'$.
\end{exercise}
\begin{hint}

    Consider the mapping  $H_0^{-s,p'}(\R^d) \to H^{s,p}_0(\R^d)'$ given by $f \mapsto \ell_f$ where
    \begin{align*}
        \ell_f(u):= \int_{\R^d}(\Lambda^s u)(\Lambda ^{-s}f).
    \end{align*}
    Show that this mapping is well defined and continuous. To see that it is invertible, show that, by duality, given $\ell \in H^{s,p}(\R^d)'$ and $u \in H^{s,p}(\R^d)$, it holds that
    \begin{align*}
        (u,\ell )=(\Lambda ^s u,\Lambda ^{-s}\ell ).
    \end{align*}
    Since $ \Lambda ^s u \in L^p(\R^d)$ we deduce that $\Lambda ^{-s}\ell \in L^{p}(\R^d)'$ and so by the Riesz representation theorem there exists $f_\ell \in L^{p'}(\R^d)$ such that $\Lambda ^{-s}\ell =\br{\cdot,f_\ell}$. Show that the mapping

    \begin{align*}
        H^{s,p}(\R^d)'                & \longrightarrow H^{-s,p'}(\R^d); \quad \ell = \br{\cdot, \Lambda^s  f_\ell} \to \Lambda^s  f_\ell,\end{align*}
    is the inverse of the previous one.
\end{hint}
\begin{exercise}
    We also know that, since $H^{s}(\R^d)$ is a Hilbert space, so by the Riesz representation theorem we have the identification $H^s(\R^d) = H^{s}(\R^d)'$. So by the previous exercise $H^{-s}(\R^d)= H^s(\R^d)$ How is this possible?
\end{exercise}
\begin{hint}
    It does \textbf{not} hold that $H^{-s}(\R^d)= H^s(\R^d)$. The problem occurs when considering too many identifications at once, as we are identifying duals using different inner products. By following the mappings we obtain a bijective isomorphism
    \begin{align*}
         & H^{s}(\R^d) \to  H^s(\R^d)' \to H^{-s}(\R^d)                                                    \\
         & u \longmapsto   \br{\cdot, u}_{H^s(\R^d)}= \br{\cdot, \Lambda^{2s} u } \mapsto \Lambda ^{2s} u.
    \end{align*}
    However, the isomorphism is $\Delta ^{2s}$, which is  hardly the identity mapping.
\end{hint}
For another example where confusion with these kind of identifications can arise see remark 3 on page  136 of \cite{brezis2011functional}.
\subsection{The dual of $H^{s,p}_0(\Omega)$}
Given an extension domain $\Omega $ and $s \in \R$ , one can define extension and restriction operators,
\begin{align*}
    E:H^{s,p}(\Omega ) \to H^{s,p}(\R^d), \quad \rho: H^{s,p}(\R^d) \to H^{s,p}(\Omega ),
\end{align*}
which verify $\rho \circ E = \Id_{H^s(\Omega )}$. As a result, restriction is surjective and we can factor $H^{s,p}(\Omega )$ as
\begin{align}\label{ismorphism}
    H^{s,p}(\Omega )\simeq H^{s,p}(\R^d)\slash H^s_{\Omega^c}(\R^d ),
\end{align}
where given a closed set $K \subset \R^d$ we define
\begin{align*}
    H^{s,p}_K(\R^d):= \set{u \in H^{s,p}(\R^d): \rm{supp}(u) \subset K},
\end{align*}
where the support is to be understood \href{https://nowheredifferentiable.com/2023-07-12-PDEs-3-Sobolev_spaces/#:~:text=Support%20of%20a%20distribution}{in the sense of distributions}
Now, given a Banach space $X$ and a closed subspace $Y \hookrightarrow X$, elements of $X'$ can be restricted to $Y$, obtaining functionals in $Y'$. The kernel of this restriction is $Y^\circ:=\set{\ell \in X': Y \subset \rm{ker}(\ell)}$. Since, by the Hahn Banach theorem, the restriction is surjective, we obtain the \href{https://math.la.asu.edu/~quigg/teach/courses/578/2008/notes/adjoints.pdf}{factorization}
\begin{align}\label{dual isomormphism}
    Y' \simeq X'\slash Y^\circ.
\end{align}
Applying this to $Y= H^{k,p}_0(\Omega )\hookrightarrow H^{k,p}(\R^d) =X$ we obtain the result of Theorem \ref{dual of integer sobolev}.
\begin{align*}
    H^{k,p}_0(\Omega )' \simeq H^{k,p}(\R^d)'\slash H^{k,p}_{\Omega^c}(\R^d)'\simeq H^{-k,p'}(\R^d)\slash H^{-k,p'}_{\Omega^c}(\R^d )\simeq H^{-k,p'}(\Omega ),
\end{align*}
where the second equality is by Exercise \ref{dual exercise} and the third by \eqref{ismorphism}.
This shows that the dual of $H^{k,p}_0(\Omega )$ is $H^{-k,p'}(\Omega )$. By also using the integer order equivalence of Theorem \ref{equivalence fractional spaces} we obtain  Theorem \ref{dual of integer sobolev}.

As a final note, if our domain has a boundary, $H_0^k(\Omega )'$ and $H^k(\Omega )'$ are not equal. Rather,
\begin{align*}
    H^{k,p}(\Omega )'\simeq H_{\overline{\Omega } }^{-k,p'}(\R^d), \quad H^{-k,p'}(\Omega ) \simeq H^{-k,p'}(\R^d)\slash H^{-k,p'}_{{\Omega }^c }(\R^d).
\end{align*}
See \cite{taylor2013partial} Section 4 for more details.

\section{Representation theorems}
We know that we can identify the spaces $H^{s,p}(\R^d )$ and $B^{s,p}(\R^d )$ with the lower order spaces $H^{s-r,p}(\R^d )$ and $B^{s-r,p}(\R^d )$ by application of $\Lambda ^r$. That is, by differentiating $r$ times. That is, spaces of lower order regularity are obtained by differentiating functions with higher regularity. We show how to extend this idea to smooth domains in some particular cases.

\begin{theorem}[Representation of $W_0^{k,p}(\Omega )'$]\label{riesz representation}
    Let $\Omega \subset \R^d$ be be open with uniformly Lipschitz boundary and let  $k \in \N$ and $p \in [1,\infty)$. Then, every element in $W^{-k,p'}(\Omega )=W^{k,p}(\Omega )'$ is the unique extension of a distribution  of the form
    \begin{align*}
        \sum_{1\leq\abs{\alpha}\leq k} D^\alpha u_\alpha\in \Dd'(\Omega ),\quad \text{where }    u_\alpha \in L^{p'}(\Omega ).
    \end{align*}

\end{theorem}
\begin{proof}
    Define the mapping
    \begin{align*}
        T: W^{k,p}(\Omega ) & \longrightarrow L^p(\Omega \to \R^n)                \\
        u                   & \longmapsto(D^\alpha u)_{1 \leq\abs{\alpha}\leq k}.
    \end{align*}
    Where the notation just says that we send $u$ to the vector formed by all its derivatives. By our definition of the norm on $W^{k,p}(\Omega )$ , we have that $T$ is an isometry and in particular continuously invertible on its image. Denote the image of $T$ by $X:=\rm{Im}(T)$. Given $\ell \in W^{-k,p'}(\Omega )$ we define
    \begin{align*}
        \ell_0: X \to \R, \quad \ell_0(\vb{w}):= \ell(T^{-1}\vb{w}), \quad \forall \vb{w} \in X.
    \end{align*}
    By Hahn Banach's theorem we can extend $\ell_0$ from $X$ to a functional $\ell_1 \in  L^p(\Omega \to \R^n)'$ and by the Riesz representation theorem we have that there exists a unique $\vb{f}=(f_\alpha)_{1\leq \abs{\alpha}\leq k }\in L^{p'}(\Omega \to \R^n)$ such that
    \begin{align*}
        \ell_1(\vb{w})=\int_{\Omega}\vb{w}\cdot \vb{h}, \quad \forall \vb{w} \in L^p(\Omega \to \R^n).
    \end{align*}
    By construction, it holds that, for all $v \in W^{k,p}(\Omega )$
    \begin{align*}
        \ell(u)=\ell_0(Tv)=\int_{\Omega}Tv\cdot \vb{f}=\sum_{1\leq\abs{\alpha}\leq k}\int_{\Omega}f_\alpha D^\alpha v .
    \end{align*}
    In particular, this holds for $v \in \Dd(\Omega )$ and if we set $u_\alpha:=(-1)^\alpha h_\alpha$ we obtain that for all $v \in \Dd(\Omega )$
    \begin{align}\label{representation}
        \ell(v)=\left(v,\sum_{1\leq\abs{\alpha}\leq k} D^\alpha u_\alpha\right)=: \omega(v)
    \end{align}
    (we recall the notation $(v,\omega)$ for the duality pairing). By definitions of the norm on $W^{k,p}(\Omega )$ and Cauchy Schwartz, we have that $\omega$ is continuous with respect to the norm on $W^{k,p}(\Omega )$ and so we may extend it uniquely to the closure of $\Dd(\Omega )$ in $W^{k,p}(\Omega )$ which is $W^{k,p}_0(\Omega )$. By \eqref{representation} the extension is necessarily $\omega$. This completes the proof.
\end{proof}
The above theorem shows that $W^{-k,p'}(\Omega )$ can be equivalently formed by differentiating $k$ times functions in $L^{p'}(\Omega )$. The proof also sheds some light as to why $W^{-s,p'}(\Omega )$ is the dual of $W^{k,p}_0(\Omega )$ and not the dual of $W^{k,p}(\Omega )$. The reason is that the elements of $W^{k,p}_0(\Omega )$ are the ones that can be extended to distributions in $\Dd'(\Omega )$ and so are the ones that we can integrate against. Finally, though the extension from $\Dd'(\Omega )$ to $W^{-s,p}(\Omega )$ is unique the functions $u_\alpha$ will not be, for example if $\abs{\alpha}>0$ it is possible to add a constant to $u_\alpha$ and still obtain the same result.
\begin{exercise}
    Show that for $s>0$ and $p \in [1,\infty)$ every element in $H^{-s,p'}(\Omega )$ can be written in the form $\restr{w}{\partial \Omega }$, where
    \begin{align*}
        w=\sum_{0\leq\abs{\alpha}\leq k} \Lambda^{\gamma } D^\alpha u_\alpha\in \Dd'(\R^d ),\quad \text{where }    u_\alpha \in L^{p'}(\R^d ),
    \end{align*}
    where $k =\left\lfloor s \right\rfloor$ and $\gamma =s-k$.
\end{exercise}
\begin{hint}
    Use that $\Lambda ^{\gamma }: H^{s,p}(\R^d) \to H^{k ,p}(\R^d)$ is an isomorphism and the just proved theorem \ref{riesz representation} together with the integer equivalence in Theorem \ref{equivalence fractional spaces} to show that
    \begin{align*}
        H^{s,p}(\R^d)' = \set{ \sum_{0\leq\abs{\alpha}\leq k} \Lambda^{\gamma } D^\alpha u_\alpha\in \Dd'(\R^d ),\quad \text{where }    u_\alpha \in L^{p'}(\R^d )}.
    \end{align*}
    Now conclude by the definition of $H^{-s,p'}(\Omega )$ for open domains \ref{bessel potential def U}.
\end{hint}
The above results extend to Besov spaces, see \cite{agranovich2015sobolev} page 227. This gives,
\begin{theorem}
    Let  $k \in \N_0, \gamma \in [0,1) \theta \in (0,1)$ and $p \in [1,\infty)$. Then,
    \begin{align*}
        B^{\theta  -k,p}(\Omega ) & = \set{ \sum_{0\leq\abs{\alpha}\leq k} D^\alpha u_\alpha\in \Dd'(\Omega ),\quad \text{where }    u_\alpha \in B^{\theta  ,p}(\Omega )} \\
        H^{\gamma -k,p}(\Omega )  & = \set{ \sum_{0\leq\abs{\alpha}\leq k} D^\alpha u_\alpha\in \Dd'(\Omega ),\quad \text{where }    u_\alpha \in H^{\gamma ,p}(\Omega )}  \\setminus	W^{\gamma -k,p}(\Omega )  & = \set{ \sum_{0\leq\abs{\alpha}\leq k} D^\alpha u_\alpha\in \Dd'(\Omega ),\quad \text{where }    u_\alpha \in W^{\gamma ,p}(\Omega )}.
    \end{align*}
\end{theorem}








\subsection{Some applications}
The following theorem can be found in \cite{agranovich2015sobolev} page 228 and serves as a generalization of the trace theorem for fractional Sobolev spaces.
\begin{theorem}[Fractional trace theorem]\label{trace theorem}
    Let $\Omega \subset \R^d$ be an open set with uniformly Lipschitz boundary. Then, for all $s>1 / p$, the trace operator $\operatorname{Tr}$ can be extended from $C_c^\infty (\R^d)$  to a bounded operator
    \begin{align*}
        \operatorname{Tr}: H^{s,p}(\Omega ) \to B^{s-1/p,p}(\partial\Omega), \quad \operatorname{Tr}: B^{s,p}(\Omega ) \to B^{s-1/p,p}(\partial\Omega).
    \end{align*}
\end{theorem}
In particular, as we have respective equalities for $W^{s,p}(\Omega )$ with $H^{s,p}(\Omega )$ and $W^{s,p}(\Omega )$ for respectively non-integer  $s$, we can also extend
\begin{align*}
    \operatorname{Tr}: W^{s,p}(\Omega ) \to B^{s-1/p,p}(\partial\Omega).
\end{align*}

One can also obtain improved Sobolev embeddings for fractional Sobolev spaces. For example, see \cite{agranovich2015sobolev} page 224
\begin{theorem}
    Given $s>n / p + \gamma $,
    where $t>0$, the space $W_p^s\left(\mathbb{R}^f\right)$ is continuously embedded in $C_b^\gamma \left(\mathbb{R}^d\right)$. This embedding also holds for $s=n / p+t$ provided that $\gamma $ is noninteger.
    In particular, $W_p^s\left(\Omega\right)$ is continuously embedded in $C_b\left(\Omega\right)$ for $s>n / p$.
\end{theorem}
\begin{theorem}[Rellich Kondratov]\label{sobolev embeddings}
    Let $\Omega \subset \R^d$ be an open set with uniformly Lipschitz boundary.
    Suppose $1<p \leq q<\infty$ and $-\infty<t \leq s<\infty$ satisfy $s-\frac{n}{p} \geq t-\frac{n}{q}$. Then,
    \begin{align*}
        W^{s, p}\left(\Omega\right) \hookrightarrow W^{t, q}\left(\overline\Omega\right)
    \end{align*}
    In particular, $W^{s, \bar{p}}\left(\Omega\right) \hookrightarrow$ $W^{t, p}\left(\Omega\right)$. Furthermore, if $\Omega $ is bounded, the inclusion is compact.

\end{theorem}

\begin{observation}
    Edit extension theorem for uniformly Lipschitz domains and then whenever $C^k$ boundary is invoked $C^{0,1}$ can be used.	The above extension result can also be proved when $\Omega$ is open with uniformly Lipschitz boundary. In practice, this just means that $\partial\Omega$ and of class $C^{0,1}$ (Lipschitz continuous). See \cite{leoni2017first} pages 423-430 for the details.
\end{observation}


\end{document}