\documentclass[12pt]{article}
\special{papersize=3in,5in}
\usepackage[utf8]{inputenc}
%PACKAGES
\usepackage{CJKutf8}
\usepackage[T1]{fontenc}
\makeatletter
\def\ps@pprintTitle{%
	\let\@oddhead\@empty
	\let\@evenhead\@empty
	\let\@oddfoot\@empty
	\let\@evenfoot\@oddfoot
}
\usepackage{amssymb,amsmath,physics,amsthm,xcolor,graphicx}
\usepackage[shortlabels]{enumitem}
% define and configure math environments
\newtheorem{theorem}{Theorem}[section]
\newtheorem{proposition}[theorem]{Proposition}
\newtheorem{corollary}[theorem]{Corollary}
\newtheorem{lemma}[theorem]{Lemma}

\newtheorem{definition}[theorem]{Definition}
\theoremstyle{definition}
\newtheorem{example}[theorem]{Example}
\newtheorem{remark}[theorem]{Remark}
\newtheorem{observation}{Observation}
\newtheorem{assumption}{Assumption}
\newtheorem{exercise}{Exercise}
\newtheorem*{hint}{Hint}
\newcommand{\red}[1]{{\color{red}#1}}
\usepackage[colorlinks=true,
	linkcolor=blue,
	filecolor=magenta,
	urlcolor=cyan,
	pdfpagemode=FullScreen,]{hyperref}

\bibliographystyle{elsarticle-num}
\newcommand{\fk}[1]{\mathfrak{#1}}\newcommand{\wh}[1]{\widehat{#1}}\renewcommand{\Im}{\mathbf{Im}}
\newcommand{\br}[1]{\left\langle#1\right\rangle} \newcommand{\set}[1]{\left\{#1\right\}} \newcommand{\qp}[1]{\left(#1\right)}\newcommand{\qb}[1]{\left[#1\right]}
\newcommand{\Id}{\mathbf{I}}\renewcommand{\ker}{\mathbf{ker}}\newcommand{\supp}[1]{\mathbf{supp}(#1)}\renewcommand{\tr}[1]{\mathrm{tr}\left(#1\right)}
\renewcommand{\norm}[1]{\left\lVert #1 \right\rVert}\renewcommand{\abs}[1]{\left| #1 \right|}
\newcommand{\U}{_}\renewcommand{\star}{*}

\newcommand{\A}{\mathbb{A}}\newcommand{\C}{\mathbb{C}}\newcommand{\E}{\mathbb{E}}\newcommand{\F}{\mathbb{F}}\newcommand{\II}{\mathbb{I}}\newcommand{\K}{\mathbb{K}}\newcommand{\LL}{\mathbb{L}}\newcommand{\M}{\mathbb{M}}\newcommand{\N}{\mathbb{N}}\newcommand{\PP}{\mathbb{P}}\newcommand{\Q}{\mathbb{Q}}\newcommand{\R}{\mathbb{R}}\newcommand{\T}{\mathbb{T}}\newcommand{\W}{\mathbb{W}}\newcommand{\Z}{\mathbb{Z}}
\newcommand{\Aa}{\mathcal{A}}\newcommand{\Bb}{\mathcal{B}}\newcommand{\Cc}{\mathcal{C}}\newcommand{\Dd}{\mathcal{D}}\newcommand{\Ee}{\mathcal{E}}\newcommand{\Ff}{\mathcal{F}}\newcommand{\Gg}{\mathcal{G}}\newcommand{\Hh}{\mathcal{H}}\newcommand{\Kk}{\mathcal{K}}\newcommand{\Ll}{\mathcal{L}}\newcommand{\Mm}{\mathcal{M}}\newcommand{\Nn}{\mathcal{N}}\newcommand{\Pp}{\mathcal{P}}\newcommand{\Qq}{\mathcal{Q}}\newcommand{\Rr}{\mathcal{R}}\newcommand{\Ss}{\mathcal{S}}\newcommand{\Tt}{\mathcal{T}}\newcommand{\Uu}{\mathcal{U}}\newcommand{\Ww}{\mathcal{W}}\newcommand{\XX}{\mathcal {X}}\newcommand{\Zz}{\mathcal{Z}}
\renewcommand{\d}{\,\mathrm{d}}
\newcommand\restr[2]{\left.#1\right|_{#2}}
\newcommand{\qt}[1]{\left(#1\right)}

\pagestyle{empty}
\setlength{\parindent}{0in}

\begin{document}
\title{Elliptic PDE I}
\author{Liam Llamazares}
\date{5/21/2023}
\maketitle
\section{ Three line summary}
\begin{itemize}
	\item Elliptic partial differential equations (PDE) are PDE with no time variable and whose leading order derivatives satisfy a positivity condition.
	\item Using Lax Milgram's theorem we can prove existence and uniqueness of weak (distributional) solutions if the transport term is zero.
	\item If the transport term is non-zero solutions still exist but they are no longer unique and are determined by the kernel of the homogeneous problem.\end{itemize}
\section{Why should I care?}
Many problems arising in physics such as the Laplace and Poisson equation are elliptic PDE. Furthermore, the tools used to analyze them can be extrapolated to other settings such as
elliptic PDE. The analysis also helps contextualize and provide motivation for theoretical tools such as Hilbert spaces, compact operators and Fredholm operators.
\section{Notation}
Given $\lambda \in \C$ we will write $\overline{\lambda }$ for the conjugate of $\lambda $. Given a subset $S$ of some topological space it is also common to write  $\overline{S}$ for the closure of $S$. Though this is a slight abuse of notation we will do the same as the meaning will always be clear from context.

Given two topological vector spaces $X,Y$ we write  $\Ll(X,Y)$ for the space of continuous linear operators from $X$ to  $Y$.

We will use the Einstein convention that indices when they are repeated are summed over. For example we will write
\begin{align*}
	\nabla \cdot (a \nabla)=\sum_{i=1}^n \partial_i a_{ij} \partial _j =\partial_i a_{ij} \partial _j.
\end{align*}
Furthermore, we will fix $U \subset \R^n$ to be an open \textbf{bounded} (we will see later why this is important) set in $\R^n$ with \textbf{no conditions} on the regularity of $\partial U$. If we need to impose regularity on the boundary we will write $\Omega$ instead of $U$.

We will use Vinogradov notation.

\section{Introduction}
Welcome back to the second post on our series of PDE. In the \href{https://nowheredifferentiable.com/2023-01-29-PDE-1/}{first post} of the series we dealt with the Fourier transform and it's application to defining spaces of weak derivatives and weak solutions to PDE. In this post we will consider an equation of the form
\begin{align}\label{PDE}
	\Ll u =f; \quad \restr{u}{\partial U} =0.
\end{align}
Where $\Ll$ is some differential operator, $f: U \to  \R$  is some known function and $u$ is the solution we want to find. We will need the following condition on $\vb{A}$  to prove well-posedness of \eqref{PDE}.
\begin{definition}
	Given $\vb{A}: U \to \R^{d \times d}, \vb{b}: U \to \R^d$ and $c:U \to \R$ we say that the differential operator
	\begin{align}\label{operator}
		\Ll:= -\nabla \cdot \vb{A} \nabla + \nabla \cdot \vb{b} +c\end{align}
	is \emph{elliptic} if there exists $\lambda >0$ such that
	\begin{align}\label{elliptic}
		\xi ^T\vb{A}(x) \xi  \geq \lambda \abs{\xi }^2 , \quad\forall \xi \in \R^d , \quad\forall x \in U .
	\end{align}
\end{definition}
There are some points to clear up. Firstly, if this is the first time you've encountered  the ellipticity condition in \eqref{elliptic} then it may seem a bit strange.  Physically speaking, in a typical \href{https://nowheredifferentiable.com/2023-12-23-PDEs-4-Physical_derivation_of_parabolic_and_elliptic_PDE/}{derivation} of our PDE in \eqref{PDE}, $u$ is the density of some substance and $\vb{A}$ corresponds to a diffusion matrix. Due to the ellipticity condition \eqref{elliptic} says that flow occurs from the region of \href{https://nowheredifferentiable.com/2023-12-23-PDEs-4-Physical_derivation_of_parabolic_and_elliptic_PDE/#:~:text=a)-,Diffusion,-%3A%20This%20is%20the}{higher to lower density}. Mathematically speaking \eqref{elliptic} will provide the necessary bound we need to apply \href{https://nowheredifferentiable.com/2023-05-30-PDE-2-Hilbert/#:~:text=degenerate.%20As%20a-,particular,-example%2C%20a%20symmetric}{Lax Milgram's theorem}.

This said, we have not yet defined which function space our coefficients live in and what $\Ll$ acts on. This is always a tricky aspect. ``How much can one get away with?''. So as to not make our lives too complicated in what remains we will make the following assumption
\begin{assumption}\label{Ass1}
	We assume that  $A_{ij}, b_i, c \in L^\infty (U)$ for all $i,j=1,\ldots,d$. Furthermore, $A$ is symmetric, that is  $A_{ij}=A_{ji}$.
\end{assumption}
T0 simplify the notation we will write for the bound on $a,\vb{b},c$
\begin{align*}
	\norm{\vb{A}}_{L^\infty(U)}+\norm{\vb{b}}_{L^\infty(U)}+\norm{c}_{L^\infty(U)}=M
\end{align*}

The first assumption will make it easy to get bounds on $\Ll$ and the second will be necessary to apply Lax Milgram's theorem and the third will prove useful when we look at the spectral theory of $\Ll$ (blogpost on this coming). Now, to make sense of \eqref{PDE} we need to define what we mean by a solution. Here the theory of Sobolev Spaces and the Fourier transform prove crucial. We will work with the following space
\begin{definition}[Negative Sobolev space]\label{dual definition}
	Given $k \in \N$ we define
	\begin{align*}
		H^{-k}(U):= H_0^k(U)'
	\end{align*}
\end{definition}
For more details on why this notation is used see the appendix \ref{dual section}.


With these assumptions we have the following
\begin{proposition}\label{domain L}
	The operator $\Ll$ defined in \eqref{operator} defines for all $s \in \R$ a bounded linear operator
	\begin{align*}
		\Ll : H_0^{s+2}\to H^s (U).
	\end{align*}
	\red{Have to add some stuff on trace and Sobolev on bounded domain.}
\end{proposition}
\begin{proof}
	We apply the usual trick of working first with a smooth functions $u$ that vanishes on the boundary. Then have that
	\begin{align*}
		\norm{\Ll u}_{H^s_0(U)}=\norm{\br{\xi }^s \wh{\Ll f}}_{L^2(U)}\lesssim  M \norm{\br{\xi }^{s+2}\wh{u}(\xi )}_{L^2(U)} =M \norm{u}_{H^{s+2}(U)} .
	\end{align*}
	Since $C_0^\infty(U)$ is dense in $H^k_0(U)$ for any $k \in \R$ we can extend $\Ll$ continuous to $H_0^{s+2}(U)$ by defining
	\begin{align*}
		\Ll u = \lim_{n \to \infty}\Ll u_n , \quad\forall u \in  H_0^{s+2}(U).
	\end{align*}
	Where $u_n \in C_0^\infty(U)$ is any sequence converging to $u$ in  $H^{s+2}(U)$.
\end{proof}
We now note that, by an integration by parts, if $u,v \in  C_0^\infty(U)$ then \begin{align*}
	\int_{U} \Ll u v =\int_{U}a \nabla u \cdot \nabla v + \int_{U} \vb{b} \nabla  u v + \int_{U} cuv=: B(u,v)   .
\end{align*}
It is clear that $B$ is bilinear in an algebraic sense. Furthermore from Cauchy Schwartz and  the fact that
\begin{align*}
	\norm{u}_{H^1(U)}\sim \norm{u}_{L^2(U)}+\norm{\nabla u}_{L^2(U\to \R^d)}.
\end{align*}
Shows that we have the bound
\begin{align}\label{cont B}
	B(u,v)\lesssim M \norm{u}_{H_0^1(U)}\norm{v}_{H_0^1(U)}.
\end{align}
This allows us as in the previous proposition to extend $B$ from $C_0^\infty(U)$ to a continuous bilinear operator on  $H^1_0(U)$. We still have not mentioned what space $f$ should be in. We just saw that it makes sense to consider $u \in  H_0^1(U)$. Since by Proposition \ref{domain L} we have $\Ll u \in H^{-1}(U)$ and since we are looking for solutions to
\begin{align*}
	\Ll u=f.
\end{align*}
We see that we should impose $f \in H^{-1}(U)$. This can all be summarized as follows.
\begin{proposition}
	Given $f \in  H^{-1}(U)$, solving equation \eqref{PDE} (under assumption \ref{Ass1}) is equivalent to finding $u \in H_0^1(U)$ such that
	\begin{align}\label{reform}
		B(u,v)= (f,v) , \quad\forall v \in  H^{1}_0(U).
	\end{align}
\end{proposition}
Where we recall the ``duality notation''
\begin{align*}
	(f,v):= f(v) .
\end{align*}




\red{Labeling assumptions probably won't work. Check it.}
As a result we have reformulated our problem to something that looks very similar to the setup of Lax Milgram's theorem. In fact, if we suppose $\vb{b}=0$ and $c \geq 0$ we are done
\begin{theorem}
	Suppose $\vb{b}=0$  and $c \geq 0$. Then given $f \in  H^{-1}(U)$ there exists a unique solution $u$ to \eqref{reform} (and thus to \eqref{PDE}). Furthermore, the solution operator $\Ll^{-1}$ is a continuous operator	 $\Ll^{-1}: H^{-1}(U) \to H^1_0(U)$ with $\norm{\Ll^{-1}} \lesssim_U \lambda ^{-1}$.
\end{theorem}
\begin{proof}
	The continuity of $B$ was proved in  \eqref{reform}. It remains to see that $B$ is coercive. This follows from the fact that for smooth $u$
	\begin{align}\label{b=0}
		B(u,u) & = \int_{U}a \nabla u \cdot \nabla u + \int_{U} cu^2 \geq \lambda \norm{\nabla u}_{L^2(U \to \R^d)} \gtrsim_U \norm{u}_{H^1_0(U)}.
	\end{align}
	Where in first inequality we used the ellipticity assumption on $\vb{A}$ and in the last inequality we used Poincaré's inequality.
\end{proof}
\red{Maybe best add $c \geq 0$ inn assumption here instead of $1$ if we later add on  $\alpha u$.}
Furthermore, by Rellich theorem $\Ll$ is compact and is self adjoint since $\vb{b}$ is  $0$ so there is a countable basis of eigenvalues in  $L^2(U)$. Furthermore they must me smooth by Prop  $2$ and Sobolev embedding.

In the previous result, we somewhat unsatisfyingly had to assume that $ \vb{b}$ was identically zero and had to impose the extra assumption  $c \geq 0$. These extra assumptions can be done away with, but at the cost of modifying our initial problem by a correction term $\gamma $ so we can once more obtain a coercive operator $B_\gamma $
\begin{theorem}[Modified problem]\label{mod}
	Given $f \in  H^{-1}(U)$ there exists some constant $\nu \geq 0$ (depending on the coefficients) such that for all $\gamma \geq \nu$  there exists a unique solution $u$ to
	\begin{align*}
		\Ll_\gamma u:= \Ll u+ \gamma u=f.
	\end{align*}
\end{theorem}
Furthermore $\Ll^{-1}: H^{-1}(U) \to H^1_0(U)$ is a continuous operator
\begin{proof}
	Once more, the proof will go through the Lax-Milgram theorem, where now we work with the bilinear operator $B_\gamma  $ associated to $\Ll_\gamma  $
	\begin{align*}
		B_\gamma  (u,v):= B(u,v) + \gamma  (u,v).
	\end{align*}
	The calculation proceeds in a similar fashion to  \eqref{b=0}, where now an additional application of Cauchy Schwartz to $\nabla u v = (\epsilon^{\frac{1}{2}} \nabla u)(\epsilon^{-\frac{1}{2}}v)$  shows that
	\begin{align*}
		B(u,u) & = \int_{U}a \nabla u \cdot \nabla u + \int_{U} \vb{b}\cdot  \nabla u v +  \int_{U} cu^2 \geq \lambda \norm{\nabla u}_{L^2(U \to \R^d)}           \\
		       & - \norm{\vb{b}}_{L^\infty(U)} \qt{\epsilon \norm{\nabla u}_{L^2(U)}+ \epsilon ^{-1}\norm{u}_{L^2(U)}}- \norm{c}_{L^\infty(U)}\norm{u}_{L^2(U)} .
	\end{align*}
	Taking $\epsilon $ small enough (smaller than $\frac{1}{2} \lambda \norm{\vb{b}}_{L^\infty(U)}^{-1}$ to be precise) and gathering up terms gives
	\begin{align}\label{b not 0}
		B(u,u) \geq \frac{\lambda}{2} \norm{\nabla u}_{L^2(U \to \R^d)} -\nu \norm{u}_{L^2(U)}.
	\end{align}
	Where we defined $\nu = \norm{\vb{b}}_{L^\infty(U)} \epsilon ^{-1}+\norm{c}_{L^\infty(U)}$.The theorem now follows from the just proved \eqref{b not 0} and Poincaré's inequality as for all $\gamma \geq \nu$
	\begin{align*}
		B_\gamma (u,u)=B(u,u)+ \gamma \norm{u}_{L^2(U)} \geq\frac{\lambda}{2} \norm{\nabla u}_{L^2(U \to \R^d)}\gtrsim _U \norm{u}_{H_0^1(U)} .
	\end{align*}
\end{proof}
Now that we proved solutions for our modified problem $\Ll_\gamma $ it would be nice if we could somehow ``unmodify''. We reason as follows: we have that $u$ solves our original problem  \eqref{PDE} if and only if
\begin{align*}
	\Ll_\gamma u - \gamma  u =f.
\end{align*}
That is, moving $\gamma  u$ over and taking the inverse of $\Ll_\gamma $, if and only if
\begin{align*}
	\gamma^{-1}u -\Ll_\gamma ^{-1} u = \Ll_\gamma ^{-1}f.
\end{align*}
By what we just proved in Proposition \ref{mod} the operator $\Ll_\gamma^{-1}$, and thus $ \gamma \Ll_\gamma ^{-1}$ is a continuous operator form $H^{-1}(U)$ to  $H^{1}(U)$, as a result it is compact and we may apply the Fredholm alternative.

\red{Might have to restrict $f$ to  $L^2$ for some adjointness....}

\appendix
\section{The dual of a Sobolev space}\label{dual section}
In this section we discuss a bit more on why the notation $H^{-s}(U)=H_0^s(U)'$ in definition \ref{dual definition}.
\subsection{The dual of $H^s(\R^d)$}
For some motivation we start by considering the case $U=\R^d$. In this case, since the closure of
$C_c^\infty(\R^d)$ in $H^s(\R^d)$ (which by definition is $H_0^s(\R^d)$) is itself $H^s(\R^d)$, we have that $H_0^s(\R^d)=H^s(\R^d)$.
\begin{exercise}[Dual of Sobolev spaces]\label{dual exercise}
	Prove the identification $H^{-s}(\R^d)=H^s(\R^d)'$.
\end{exercise}
\begin{hint}

	Consider the mapping  $H_0^{-s}(\R^d) \to H^s_0(\R^d)'$ given by $f \mapsto \ell_f$ where
	\begin{align*}
		\ell_f(u):= \int_{\R^d}\wh{u} \wh{f} .
	\end{align*}
	Show that this mapping is well defined and continuous. Then, use the Riesz representation theorem to show that it is surjective with inverse given by the mapping
	\begin{align*}
		H^s(\R^d)'                & \longrightarrow H^{-s}(\R^d)                                 \\
		\ell = \br{\cdot, g_\ell} & \longmapsto \mathcal{F}^{-1}(\br{\xi}^{2s}\wh{g_\ell}(\xi ))
		.\end{align*}


	and show that it is an isomorphism.
\end{hint}
\begin{exercise}
	We also know that, since $H^s(\R^d)$ is a Hilbert space, so by the Riesz representation theorem we have the identification $H^s(\R^d) = H^{s}_0(\R^d)'$. So by the previous exercise $H^{-s}(\R^d)= H^s(\R^d)$ How is this possible?
\end{exercise}
\begin{hint}
	In fact it does \textbf{not} hold that $H^{-s}(\R^d)= H^s(\R^d)$. The problem occurs when considering too many identifications at once. By following the mappings we obtain a bijective isomorphism
	\begin{align*}
		H^{s}(\R^d) \to & H^s(\R^d)' \to H^{-s}(\R^d)                                                              \\
		u \longmapsto   & \br{\cdot, u}_{H^s(\R^d)} \mapsto \mathcal{F}^{-1}\left(\br{\xi}^{2s}\wh{u}(\xi )\right)
	\end{align*}
	Though this is an isomorphism it is hardly the identity mapping.
\end{hint}
For another example where confusion with these kind of identifications can arise see Remark 3 on page  136 of \cite{brezis2011functional}.

\subsection{The dual of $H^k_0(\Omega)$}
Here things get a little bit more complicated, mainly because we have not yet defined what is meant by
$H^s(\Omega)$ when $s$ is not a positive integer. So a priori $H^{-k}(\Omega)$ makes no sense. However, one can use complex interpolation to define such spaces. This done, one can show that if we embed $\Omega$ in a smooth manifold (without boundary) $M$, then $H_0^k(M)'=H^{-k}(M)$ and we can define extension and restriction operators
\begin{align*}
	E:H^s(\Omega ) \to H^s(M), \quad \rho: H^s(M) \to H^s(\Omega ),
\end{align*}
which verify $\rho \circ E = \Id_{H^s(\Omega )}$. As a result, restriction is surjective and we can factor $H^s(\Omega )$ as
\begin{align}\label{ismorphism}
	H^s(\Omega )\simeq H^s(M)\slash H^s_K(M ), \quad K=\overline{M\setminus \Omega }.
\end{align}
Where given a closed set $B\subset M$ we define
\begin{align*}
	H^s_B(M):= \set{u \in H^s(M): \supp{u} \subset B}.
\end{align*}
Now, given a Banach space $X$ and a closed subspace $Y \hookrightarrow X$ it holds that elements of $X'$ can be restricted to $Y'$. The kernel of this restriction is $Y^\circ$, and as a result we obtain the \href{https://math.la.asu.edu/~quigg/teach/courses/578/2008/notes/adjoints.pdf}{factorization}
\begin{align}\label{dual isomormphism}
	Y' \simeq X'\slash Y^\circ, \quad Y^\circ:= \set{\ell \in X': Y \subset \rm{ker}(\ell)}.
\end{align}
Applying this to $Y= H^k_0(\Omega )\hookrightarrow H^k(M) =X$ we obtain
\begin{align*}
	H^{k}_0(\Omega )' \simeq H^{k}(M)'\slash H^{k}_K(M)'\simeq H^{-k}(M)\slash H^{-k}_K(M )\simeq H^{-k}(\Omega ).
\end{align*}
Which justifies the notation in Definition \ref{dual definition}. The reason why we can only consider
integer $k$ is that, for general $s \in \R$, and somewhat surprisingly given the integer case, it does not hold that $H^s_0(\Omega )$ are the distributions with support in $\bar{\Omega}$. That is, in general
\begin{align*}
	H^s_0(\Omega )\neq H^{s}_{\overline{\Omega } }(M).
\end{align*}
Though, by continuity of the trace, given  $ u \in H^s_0(\Omega )$ for $s>1/2$ we have that $\restr{u}{\partial \Omega } =0$.
As a final note, since in this case our domain has a boundary, $H_0^k(\Omega )'$ and $H^k(\Omega )'$ are not equal. Rather,
\begin{align*}
	H^s(\Omega )'\simeq H_{\overline{\Omega } }^{-s}(M), \quad H^{-k}(\Omega ) \simeq H^{-k}(\Omega )\slash H_{\partial \Omega }^{-k}(\Omega ).
\end{align*}
Where the second isomorphism can be seen by taking $X=H^k(\Omega )$ and $Y=H^k_{0 }(\Omega )$ in \eqref{dual isomormphism}. See \cite{taylor2013partial} Section 4 for more details.
\section{Fractional Sobolev spaces and  some generalizations}
In this section we discuss how to define fractional Sobolev spaces and some generalizations of the theory to cover more general domains. We begin with the following definition.
\begin{definition}[Fractional Sobolev spaces]
	Let $\gamma  \in (0,1), p \in [1,\infty)$ and $\Omega  \subset \R^n$ be an arbitrary open set we define
	\begin{align*}
		W^{\gamma  ,p}(U):= \set{u \in L^p(U): | u |_{\gamma,p}<\infty},
	\end{align*}
	where
	\begin{align}\label{fractional seminorm}
		| u |_{\gamma  ,p}:= \left(\int_{U}\int_{U}\frac{\abs{u(x+y)-u(x)}^p}{\abs{y}^{n+\gamma p}}\d x \d y\right)^{\frac{1}{p}},
	\end{align}
	and we give it the norm
	\begin{align*}
		\norm{u}_{W^{\gamma,p}(U)}:= \left(\norm{u}_{L^p(U)}^p+| u |_{\gamma ,p}^p\right)^{\frac{1}{p}}.
	\end{align*}
	Now, given $k \in \N$ and $s:=k+\gamma $, we define
	\begin{align*}
		W^{s ,p}(U):= \set{u \in W^{k ,p}(U): \nabla^k  u \in W^{\gamma ,p}(U \to \R^{d^k})}.
	\end{align*}
	And give it the norm
	\begin{align}\label{norm def}
		\norm{u}_{W^{s,p}(U)}:= \left(\norm{u}_{W^{k,p}(U)}^p+ \sum_{\abs{\alpha}=k }\norm{D^\alpha u}_{W^{\gamma ,p}(U)}^p\right)^\frac{1}{p}
	\end{align}
\end{definition}
The above definition mimics that of the H\"older spaces, with the addition that we now require integrability. The factor $\abs{x-y}^{n+\gamma p}$ is chosen so that the norm is scale invariant. That is, if we replace $U$ by $\lambda U$, it holds that $|u|_{s,p}$ by $|u(\lambda \cdot )|_{s,p}$.
\begin{exercise}
	Show that $W^{s,p}(U)$ is a Banach space.
\end{exercise}
\begin{hint}
	To show that $| \cdot |_{s,p}$ is a norm apply Minkowski's inequality to $u$ and to $f_u(x,y):=(u(x)-u(y))/(x-y)^{n/p+s}$. Given a Cauchy sequence show that, since $L^p(U)$ is complete, $u_n \to u$ in $L^p(U)$ and that $f_{u_n} \to f_u$ in $L^p(U\times U)$ to conclude that $u_n \to u$ in $W^{s,p}(U)$.
\end{hint}
Though these spaces can be defined for any open set $U$, they are most useful when $U=\R^d$ or $U$  is bounded open and Lipschitz (that is $U$ is of class $C^{0,1}$). This is because of the following result.
\begin{proposition}[Inclusion ordered by regularity]
	Let  $\Omega \subset \R^d$ be bounded open and Lipschitz. Then, for $p \in [1,\infty)$ and $0<s<s'$ it holds that
	\begin{align*}
		W^{s',p}(\Omega )\hookrightarrow W^{s,p}(\Omega ), \quad W^{s',p}(\R^d)\hookrightarrow W^{s,p}(\R^d).
	\end{align*}
\end{proposition}
The proof can be found in \cite{di2012hitchhiker's} page 10. The regularity of the domain is necessary to be able to extend functions in $W^{1,p}(\Omega )$ to $W^{1,p}(\R^d)$. The result is not true otherwise and an example is given in this same reference.

In our previous post on SObolev spaces and the Fourier transform we already gave a definition of fractional Sobolev spaces in the particular case $\Omega =\R^d$ and $p=2$. For consistency, it is necessary to show that both are equal. This is done in the following exercise.
\begin{exercise}[Equivalence of fractional spaces]\label{equivalence fractional spaces}
	Show that
	\begin{align*}
		W^{s,2}(\R^d)= H^s(\R^d).
	\end{align*}
\end{exercise}
\begin{hint}
	We want to show that the norms are equivalent. That is, that
	\begin{align*}
		\norm{u}_{W^{s,2}(\R^d)}\sim \norm{u}_{H^s(\R^d)}.
	\end{align*}
	We already know this is the case when $s$ is an integer so it suffices to show that the norms are equivalent for $s= \gamma  \in (0,1)$. That is, that
	\begin{align*}
		|u|_{s,2}^2\sim \int_{\mathbb{R}^d}|\xi|^{2 s}|\mathcal{F} u(\xi)|^2 d \xi
	\end{align*}
	By multiple changes of variable and Plancherel's theorem we have that
	\begin{align*}
		 & |u|_{\gamma ,2}^2  =\int_{\R^d}\int_{\R^d}\frac{\abs{u(x+y)-u(y)}^2}{\abs{x}^{d+2\gamma	}}\d x \d y                                                                                                       = \int_{\R^d}\frac{\norm{\Ff (u(x+\cdot )-u)}^2}{\abs{x}^{d+2\gamma	}}\d x \\
		 & =\int_{\R^d}\int_{\R^d}  \frac{|e^{-2 \pi i x \cdot \xi}-1|^2}{\abs{x}^{d+2\gamma	}}|\wh{u}(\xi)|^2\d x\d\xi =\int_{\R^d}\left(\int_{\R^d}  \frac{1-\cos(2\pi \xi\cdot x)}{\abs{x}^{d+2\gamma	}}\d x\right)|\wh{u}(\xi)|^2\d\xi.
	\end{align*}
	To treat the inner integral we note that it is rotationally invariant and so, by rotating, to the first axis and later changing variable $x \to x / \abs{\xi}$ we get
	\begin{align*}
		\int_{\R^d}  \frac{1-\cos(2\pi \xi\cdot x)}{\abs{x}^{d+2\gamma	}}\d x & =\int_{\R^d}  \frac{1-\cos(2\pi \abs{\xi}x_1 )}{\abs{x}^{d+2\gamma	}}\d x                                          \\
		                                                                      & =\abs{\xi}^{2 \gamma } \int_{\R^d}  \frac{1-\cos(2\pi  x_1) }{\abs{x}^{d+2\gamma	}}\d x\sim \abs{\xi}^{2 \gamma }.
	\end{align*}
	The last integral is finite as, since $d+2\gamma >d$, the tails $\abs{\xi}\to\infty$ are controlled, and since $1-\cos(2\pi x_1)\sim x_1^2\leq \abs{x}^2$ the integrand has order $-d+2(1-\gamma)>-d$ for $\abs{\xi}\sim 0$ . That said, substituting this back into the previous expression gives the desired result.
\end{hint}
The above suggests that integrals of the kind in \eqref{fractional seminorm} can corresponding to differentiating a fractional amount of times. This indeed is the case
\begin{definition}
	Given $\gamma  \in [0,+\infty)$ and $u \in \Ss (\R^d)$ we define the fractional Laplacian as
	\begin{align*}
		(-\Delta )^{\gamma }u(x):= \mathcal{F}^{-1}(\abs{2\pi\xi}^{2\gamma }\wh{u}(\xi )).
	\end{align*}
\end{definition}
\begin{proposition}
	For $\gamma  \in (0,1)$ and $u \in H^s(\R^d)$ it holds that
	\begin{align*}
		(-\Delta )^{\gamma }u(x)=-C\int_{\R^d}\frac{u(x+y)-u(y)}{\abs{x}^{d+2\gamma}}\d y,
	\end{align*}
	where $C$ is a constant that depends on $d,\gamma $.
\end{proposition}
\begin{proof}
	The above equality may seem odd at first if we compare with the integral in \eqref{fractional seminorm} where a square appears in the numerator which gives us our $2$  in the $2 \gamma $. However, it is justified by the fact that, by the change of variables $y \to -y$,
	\begin{align*}
		\int_{\R^d}\frac{u(x+y)-u(y)}{\abs{x}^{d+2\gamma}}\d x=\int_{\R^d}\frac{u(x-y)-u(y)}{\abs{x}^{d+2\gamma}}\d x.
	\end{align*}
	So we can get the \emph{second} order difference in the numerator by adding the two integrals.
	\begin{align}\label{second order}
		\int_{\R^d}\frac{u(x+y)-u(y)}{\abs{x}^{d+2\gamma}}\d x=\frac{1}{2}\int_{\R^d}\frac{u(x+y)-2u(y)+u(x-y)}{\abs{x}^{d+2\gamma}}\d x.
	\end{align}
	That said, we must show that
	\begin{align*}
		\abs{\xi}^{2\gamma }\wh{u}(\xi )\sim -\Ff \left(\int_{\R^d}\frac{u(x+y)-u(y)}{\abs{x}^{d+2\gamma}}\d x\right)
	\end{align*}
	Using \eqref{second order} and proceeding as in exercise \ref{equivalence fractional spaces} gives
	\begin{align*}
		 & -\Ff \left(\int_{\R^d}\frac{u(x+y)-u(y)}{\abs{x}^{d+2\gamma}}\d x\right)= -\frac{1}{2} \int_{\R^d}\left(\int_{\R^d}\frac{e^{-2\pi i x \cdot \xi}-2+e^{2\pi i x \cdot \xi}}{\abs{x}^{d+2\gamma}}\d x\right) \wh{u}(\xi)\d \xi      \\
		 & =\int_{\R^d}\left(\int_{\R^d}\frac{1-\cos(2\pi x \cdot \xi)}{\abs{x}^{d+2\gamma}}\d x\right) \wh{u}(\xi)\d \xi =\int_{\R^d}  \frac{1-\cos(2\pi  x_1) }{\abs{x}^{d+2\gamma	}}\d x\int_{\R^d}\abs{\xi}^{2 \gamma }\wh{u}(\xi)\d \xi \\& \sim \abs{\xi}^{2 \gamma }\wh{u}(\xi)\d \xi.
	\end{align*}
	This completes the proof, and shows that the explicit expression for $C$ is
	\begin{align*}
		C=\frac{1}{(2\pi)^{2 \gamma }}\int_{\R^d}  \frac{1-\cos(2\pi  x_1) }{\abs{x}^{d+2\gamma	}}\d x.
	\end{align*}
\end{proof}
We now have a description of fractional Sobolev spaces for all $s>0$ and have seen its connection to the fractional Laplacian and the Fourier transform when the integrability index is $p=2$. We now discuss how to extend our study to negative $s$. Thus, covering the whole real line $s \in \R$.

\begin{definition}\label{dual definition}
	Given $s>0, p \in [1,\infty)$ and a bounded Lipschitz domain $\Omega $ we define
	\begin{align*}
		W^{-s,p'}(\Omega ):= W^{s,p}_0(\Omega )',
	\end{align*}
	where $p'$ is the conjugate exponent of $p$.
\end{definition}
This definition seems a bit arbitrary, why should we define $W^{-s,p'}(\Omega )$ as the dual of $W^{s,p}_0(\Omega )$? Why not choose it to be the dual of $W^{s,p}(\Omega )$, and where does the dual appear from in the first place. The following representation theorem provides the answer to these questions.
\begin{theorem}[Riesz representation for $W_0^{s,p}(\Omega )$]\label{riesz representation}
	Let $\Omega \subset \R^d$ be a bounded Lipschitz domain or equal to $\R^d$. Given,  $s>0$ and $p \in [1,\infty)$ write $k=\left\lfloor  s\right\rfloor$ and $\gamma = s- \lfloor s\rfloor$ so $s= k+\gamma $. Then,  every element in $W^{-s,p'}(\Omega )$ is the unique extension of a distribution  of the form
	\begin{align*}
		\sum_{1\leq\abs{\alpha}\leq k}\left( D^\alpha f_\alpha+  D^\alpha (-\Delta )^{\gamma /2} g_\alpha\right)\in \Dd'(\Omega ),\quad \text{where }    f_\alpha, g_\alpha \in L^{p'}(\Omega ).
	\end{align*}

\end{theorem}
\begin{proof}
	Define the mapping
	\begin{align*}
		T: W^{s,p}(\Omega ) & \longrightarrow L^p(\Omega \to \R^n)                                                   \\
		u                   & \longmapsto(D^\alpha u, D^\alpha (-\Delta )^{\gamma /2} u)_{1 \leq\abs{\alpha}\leq k}.
	\end{align*}
	Where the notation just says that we send $u$ to the vector formed by all its derivatives. By our definition of the norm on $W^{s,p}(\Omega )$ \eqref{norm def}, we have that $T$ is an isometry and in particular continuously invertible on its image. Denote the image of $T$ by $X:=\Im(T)$. Given $\ell \in W^{-s,p'}(\Omega )$ we define
	\begin{align*}
		\ell_0: X \to \R, \quad \ell_0(\vb{w}):= \ell(T^{-1}\vb{w}), \quad \forall \vb{w} \in X.
	\end{align*}
	By Hahn Banach's theorem we can extend $\ell_0$ from $X$ to a functional $\ell_1 \in  L^p(\Omega \to \R^n)'$ and by the Riesz representation theorem we have that there exists a unique $\vb{h}=(a_\alpha, b_\alpha)_{1\leq \abs{\alpha}\leq k }\in L^{p'}(\Omega \to \R^n)$ such that
	\begin{align*}
		\ell_1(\vb{w})=\int_{\Omega}\vb{w}\cdot \vb{h}, \quad \forall \vb{w} \in L^p(\Omega \to \R^n).
	\end{align*}
	By construction, it holds that, for all $u \in W^{s,p}(\Omega )$
	\begin{align*}
		\ell(u)=\ell_0(Tu)=\int_{\Omega}Tu\cdot \vb{h}=\sum_{1\leq\abs{\alpha}\leq k}\int_{\Omega}a_\alpha D^\alpha u  +b_\alpha D^\alpha (-\Delta )^{\gamma /2}u.
	\end{align*}
	In particular, this holds for all $u \in \Dd(\Omega )$ and if we set $f_\alpha:=(-1)^\alpha a_\alpha$ and $g_\alpha:=(-1)^\alpha b_\alpha$ we obtain that for all $u \in \Dd(\Omega )$
	\begin{align}\label{representation}
		\ell(u)=\left(u,\sum_{1\leq\abs{\alpha}\leq k} D^\alpha u_\alpha+ \sum_{1\leq\abs{\alpha}\leq k} D^\alpha (-\Delta )^{\gamma /2}v_\alpha\right)=: \omega(u)
	\end{align}
	(we recall the notation $(u,\omega)$ for the duality pairing). By definitions of the norm on $W^{s,p}(\Omega )$ and Cauchy Schwartz, we have that $\omega$ is continuous with respect to the norm on $W^{s,p}(\Omega )$ and so we may extend it uniquely to the closure of $\Dd(\Omega )$ in $W^{s,p}(\Omega )$ which is $W^{s,p}_0(\Omega )$. By \eqref{representation} the extension is necessarily $\omega$. This completes the proof.
\end{proof}
The above theorem shows that $W^{-s,p'}(\Omega )$ can be equivalently formed by differentiating $s$ times functions in $L^{p'}(\Omega )$ and is more natural than the definition in \eqref{dual definition}. The proof also sheds some light as to why we define $W^{-s,p'}(\Omega )$ as the dual of $W^{s,p}_0(\Omega )$ and not as the dual of $W^{s,p}(\Omega )$. The reason is that the elements of $W^{s,p}_0(\Omega )$ are the ones that can be extended to distributions in $\Dd'(\Omega )$ and so are the ones that we can integrate against. Finally, though the extension from $\Dd'(\Omega )$ to $W^{-s,p}(\Omega )$ is unique the functions $f_\alpha, g_\alpha$ will not be, for example if $\abs{\alpha}>0$ it is possible to add a constant to $f_\alpha$ and $g_\alpha$ and still obtain the same result.

\begin{observation}
	Edit extension theorem for uniformly Lipschitz domains and then whenever $C^k$ boundary is invoked $C^{0,1}$ can be used.	The above extension result can also be proved when $\Omega$ is uniformly Lipschitz. In practice, this just means that $\partial\Omega$ and of class $C^{0,1}$ (Lipschitz continuous). See \cite{leoni2017first} pages 423-430 for the details.
\end{observation}









\bibliography{biblio.bib}
\end{document}
